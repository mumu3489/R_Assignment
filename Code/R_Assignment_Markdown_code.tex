\documentclass[]{article}
\usepackage{lmodern}
\usepackage{amssymb,amsmath}
\usepackage{ifxetex,ifluatex}
\usepackage{fixltx2e} % provides \textsubscript
\ifnum 0\ifxetex 1\fi\ifluatex 1\fi=0 % if pdftex
  \usepackage[T1]{fontenc}
  \usepackage[utf8]{inputenc}
\else % if luatex or xelatex
  \ifxetex
    \usepackage{mathspec}
  \else
    \usepackage{fontspec}
  \fi
  \defaultfontfeatures{Ligatures=TeX,Scale=MatchLowercase}
\fi
% use upquote if available, for straight quotes in verbatim environments
\IfFileExists{upquote.sty}{\usepackage{upquote}}{}
% use microtype if available
\IfFileExists{microtype.sty}{%
\usepackage{microtype}
\UseMicrotypeSet[protrusion]{basicmath} % disable protrusion for tt fonts
}{}
\usepackage[margin=1in]{geometry}
\usepackage{hyperref}
\hypersetup{unicode=true,
            pdftitle={R-Assignment\_Markdown},
            pdfauthor={Qi Mu},
            pdfborder={0 0 0},
            breaklinks=true}
\urlstyle{same}  % don't use monospace font for urls
\usepackage{color}
\usepackage{fancyvrb}
\newcommand{\VerbBar}{|}
\newcommand{\VERB}{\Verb[commandchars=\\\{\}]}
\DefineVerbatimEnvironment{Highlighting}{Verbatim}{commandchars=\\\{\}}
% Add ',fontsize=\small' for more characters per line
\usepackage{framed}
\definecolor{shadecolor}{RGB}{248,248,248}
\newenvironment{Shaded}{\begin{snugshade}}{\end{snugshade}}
\newcommand{\KeywordTok}[1]{\textcolor[rgb]{0.13,0.29,0.53}{\textbf{#1}}}
\newcommand{\DataTypeTok}[1]{\textcolor[rgb]{0.13,0.29,0.53}{#1}}
\newcommand{\DecValTok}[1]{\textcolor[rgb]{0.00,0.00,0.81}{#1}}
\newcommand{\BaseNTok}[1]{\textcolor[rgb]{0.00,0.00,0.81}{#1}}
\newcommand{\FloatTok}[1]{\textcolor[rgb]{0.00,0.00,0.81}{#1}}
\newcommand{\ConstantTok}[1]{\textcolor[rgb]{0.00,0.00,0.00}{#1}}
\newcommand{\CharTok}[1]{\textcolor[rgb]{0.31,0.60,0.02}{#1}}
\newcommand{\SpecialCharTok}[1]{\textcolor[rgb]{0.00,0.00,0.00}{#1}}
\newcommand{\StringTok}[1]{\textcolor[rgb]{0.31,0.60,0.02}{#1}}
\newcommand{\VerbatimStringTok}[1]{\textcolor[rgb]{0.31,0.60,0.02}{#1}}
\newcommand{\SpecialStringTok}[1]{\textcolor[rgb]{0.31,0.60,0.02}{#1}}
\newcommand{\ImportTok}[1]{#1}
\newcommand{\CommentTok}[1]{\textcolor[rgb]{0.56,0.35,0.01}{\textit{#1}}}
\newcommand{\DocumentationTok}[1]{\textcolor[rgb]{0.56,0.35,0.01}{\textbf{\textit{#1}}}}
\newcommand{\AnnotationTok}[1]{\textcolor[rgb]{0.56,0.35,0.01}{\textbf{\textit{#1}}}}
\newcommand{\CommentVarTok}[1]{\textcolor[rgb]{0.56,0.35,0.01}{\textbf{\textit{#1}}}}
\newcommand{\OtherTok}[1]{\textcolor[rgb]{0.56,0.35,0.01}{#1}}
\newcommand{\FunctionTok}[1]{\textcolor[rgb]{0.00,0.00,0.00}{#1}}
\newcommand{\VariableTok}[1]{\textcolor[rgb]{0.00,0.00,0.00}{#1}}
\newcommand{\ControlFlowTok}[1]{\textcolor[rgb]{0.13,0.29,0.53}{\textbf{#1}}}
\newcommand{\OperatorTok}[1]{\textcolor[rgb]{0.81,0.36,0.00}{\textbf{#1}}}
\newcommand{\BuiltInTok}[1]{#1}
\newcommand{\ExtensionTok}[1]{#1}
\newcommand{\PreprocessorTok}[1]{\textcolor[rgb]{0.56,0.35,0.01}{\textit{#1}}}
\newcommand{\AttributeTok}[1]{\textcolor[rgb]{0.77,0.63,0.00}{#1}}
\newcommand{\RegionMarkerTok}[1]{#1}
\newcommand{\InformationTok}[1]{\textcolor[rgb]{0.56,0.35,0.01}{\textbf{\textit{#1}}}}
\newcommand{\WarningTok}[1]{\textcolor[rgb]{0.56,0.35,0.01}{\textbf{\textit{#1}}}}
\newcommand{\AlertTok}[1]{\textcolor[rgb]{0.94,0.16,0.16}{#1}}
\newcommand{\ErrorTok}[1]{\textcolor[rgb]{0.64,0.00,0.00}{\textbf{#1}}}
\newcommand{\NormalTok}[1]{#1}
\usepackage{graphicx,grffile}
\makeatletter
\def\maxwidth{\ifdim\Gin@nat@width>\linewidth\linewidth\else\Gin@nat@width\fi}
\def\maxheight{\ifdim\Gin@nat@height>\textheight\textheight\else\Gin@nat@height\fi}
\makeatother
% Scale images if necessary, so that they will not overflow the page
% margins by default, and it is still possible to overwrite the defaults
% using explicit options in \includegraphics[width, height, ...]{}
\setkeys{Gin}{width=\maxwidth,height=\maxheight,keepaspectratio}
\IfFileExists{parskip.sty}{%
\usepackage{parskip}
}{% else
\setlength{\parindent}{0pt}
\setlength{\parskip}{6pt plus 2pt minus 1pt}
}
\setlength{\emergencystretch}{3em}  % prevent overfull lines
\providecommand{\tightlist}{%
  \setlength{\itemsep}{0pt}\setlength{\parskip}{0pt}}
\setcounter{secnumdepth}{0}
% Redefines (sub)paragraphs to behave more like sections
\ifx\paragraph\undefined\else
\let\oldparagraph\paragraph
\renewcommand{\paragraph}[1]{\oldparagraph{#1}\mbox{}}
\fi
\ifx\subparagraph\undefined\else
\let\oldsubparagraph\subparagraph
\renewcommand{\subparagraph}[1]{\oldsubparagraph{#1}\mbox{}}
\fi

%%% Use protect on footnotes to avoid problems with footnotes in titles
\let\rmarkdownfootnote\footnote%
\def\footnote{\protect\rmarkdownfootnote}

%%% Change title format to be more compact
\usepackage{titling}

% Create subtitle command for use in maketitle
\newcommand{\subtitle}[1]{
  \posttitle{
    \begin{center}\large#1\end{center}
    }
}

\setlength{\droptitle}{-2em}

  \title{R-Assignment\_Markdown}
    \pretitle{\vspace{\droptitle}\centering\huge}
  \posttitle{\par}
    \author{Qi Mu}
    \preauthor{\centering\large\emph}
  \postauthor{\par}
      \predate{\centering\large\emph}
  \postdate{\par}
    \date{October 8, 2018}


\begin{document}
\maketitle

\subsection{Import data}\label{import-data}

The project structure is R-Assignment Code: store the script for this
project Data: store the orginal (input) files Instruction: store the
assignment instruction Output: store all the output files First load
date from local folder. \texttt{../Data/file} was used because that is
where I stored my input files.\texttt{../} represent going up a level,
similar to unix system. *The web address based loading data does not
work here.

\begin{Shaded}
\begin{Highlighting}[]
\KeywordTok{library}\NormalTok{(tidyverse)}
\end{Highlighting}
\end{Shaded}

\begin{verbatim}
## Warning: package 'tidyverse' was built under R version 3.5.1
\end{verbatim}

\begin{verbatim}
## -- Attaching packages ---------------------------------------------------------- tidyverse 1.2.1 --
\end{verbatim}

\begin{verbatim}
## v ggplot2 2.2.1     v purrr   0.2.5
## v tibble  1.4.2     v dplyr   0.7.6
## v tidyr   0.8.1     v stringr 1.3.0
## v readr   1.1.1     v forcats 0.3.0
\end{verbatim}

\begin{verbatim}
## Warning: package 'dplyr' was built under R version 3.5.1
\end{verbatim}

\begin{verbatim}
## -- Conflicts ------------------------------------------------------------- tidyverse_conflicts() --
## x dplyr::filter() masks stats::filter()
## x dplyr::lag()    masks stats::lag()
\end{verbatim}

\begin{Shaded}
\begin{Highlighting}[]
\NormalTok{Geno <-}\StringTok{ }\KeywordTok{read_tsv}\NormalTok{(}\StringTok{"../Data/fang_et_al_genotypes.txt"}\NormalTok{, }\DataTypeTok{col_names =}\NormalTok{ T )}
\end{Highlighting}
\end{Shaded}

\begin{verbatim}
## Parsed with column specification:
## cols(
##   .default = col_character()
## )
\end{verbatim}

\begin{verbatim}
## See spec(...) for full column specifications.
\end{verbatim}

\begin{Shaded}
\begin{Highlighting}[]
\NormalTok{SNP <-}\StringTok{ }\KeywordTok{read_tsv}\NormalTok{(}\StringTok{"../Data/snp_position.txt"}\NormalTok{)}
\end{Highlighting}
\end{Shaded}

\begin{verbatim}
## Parsed with column specification:
## cols(
##   SNP_ID = col_character(),
##   cdv_marker_id = col_integer(),
##   Chromosome = col_character(),
##   Position = col_character(),
##   alt_pos = col_character(),
##   mult_positions = col_character(),
##   amplicon = col_character(),
##   cdv_map_feature.name = col_character(),
##   gene = col_character(),
##   `candidate/random` = col_character(),
##   Genaissance_daa_id = col_integer(),
##   Sequenom_daa_id = col_integer(),
##   count_amplicons = col_integer(),
##   count_cmf = col_integer(),
##   count_gene = col_integer()
## )
\end{verbatim}

\subsubsection{Part I}\label{part-i}

\subsection{Data inspection}\label{data-inspection}

There are many ways to inspect the data. Below is a list of functions we
can do. However, I masked certain fucntions because they will produce
too large outputs. These actions can be done for smaller sized
dateframes. \texttt{head()} is showing the similiar results as by
looking at the data itself (both are showing the tibble of the first few
rows), so I masked it as well.

\begin{Shaded}
\begin{Highlighting}[]
\NormalTok{Geno}
\end{Highlighting}
\end{Shaded}

\begin{verbatim}
## # A tibble: 2,782 x 986
##    Sample_ID JG_OTU  Group abph1.20 abph1.22 ae1.3 ae1.4 ae1.5 an1.4 ba1.6
##    <chr>     <chr>   <chr> <chr>    <chr>    <chr> <chr> <chr> <chr> <chr>
##  1 SL-15     T-aust~ TRIPS ?/?      ?/?      T/T   G/G   T/T   C/C   ?/?  
##  2 SL-16     T-aust~ TRIPS ?/?      ?/?      T/T   ?/?   T/T   C/C   A/G  
##  3 SL-11     T-brav~ TRIPS ?/?      ?/?      T/T   G/G   T/T   ?/?   G/G  
##  4 SL-12     T-brav~ TRIPS ?/?      ?/?      T/T   G/G   T/T   C/C   G/G  
##  5 SL-18     T-cund  TRIPS ?/?      ?/?      T/T   G/G   T/T   C/C   ?/?  
##  6 SL-2      T-dact~ TRIPS ?/?      ?/?      T/T   G/G   T/T   C/C   A/G  
##  7 SL-4      T-dact~ TRIPS ?/?      ?/?      T/T   G/G   T/T   C/C   G/G  
##  8 SL-6      T-dact~ TRIPS ?/?      ?/?      T/T   G/G   T/T   C/C   ?/?  
##  9 SL-7      T-dact~ TRIPS ?/?      ?/?      T/T   G/G   T/T   C/C   G/G  
## 10 SL-5      T-dact~ TRIPS ?/?      ?/?      T/T   G/G   T/T   C/C   ?/?  
## # ... with 2,772 more rows, and 976 more variables: ba1.9 <chr>,
## #   bt2.5 <chr>, bt2.7 <chr>, bt2.8 <chr>, Fea2.1 <chr>, Fea2.5 <chr>,
## #   id1.3 <chr>, lg2.11 <chr>, lg2.2 <chr>, pbf1.1 <chr>, pbf1.2 <chr>,
## #   pbf1.3 <chr>, pbf1.5 <chr>, pbf1.6 <chr>, pbf1.7 <chr>, pbf1.8 <chr>,
## #   PZA00003.11 <chr>, PZA00004.2 <chr>, PZA00005.8 <chr>,
## #   PZA00005.9 <chr>, PZA00006.13 <chr>, PZA00006.14 <chr>,
## #   PZA00008.1 <chr>, PZA00010.5 <chr>, PZA00013.10 <chr>,
## #   PZA00013.11 <chr>, PZA00013.9 <chr>, PZA00015.4 <chr>,
## #   PZA00017.1 <chr>, PZA00018.5 <chr>, PZA00029.11 <chr>,
## #   PZA00029.12 <chr>, PZA00030.11 <chr>, PZA00031.5 <chr>,
## #   PZA00041.3 <chr>, PZA00042.2 <chr>, PZA00042.5 <chr>,
## #   PZA00043.7 <chr>, PZA00045.1 <chr>, PZA00047.2 <chr>,
## #   PZA00049.12 <chr>, PZA00050.9 <chr>, PZA00051.2 <chr>,
## #   PZA00058.5 <chr>, PZA00058.6 <chr>, PZA00060.2 <chr>,
## #   PZA00061.1 <chr>, PZA00065.2 <chr>, PZA00069.4 <chr>,
## #   PZA00070.5 <chr>, PZA00078.2 <chr>, PZA00079.1 <chr>,
## #   PZA00081.17 <chr>, PZA00084.2 <chr>, PZA00084.3 <chr>,
## #   PZA00086.8 <chr>, PZA00088.3 <chr>, PZA00090.2 <chr>,
## #   PZA00092.1 <chr>, PZA00092.5 <chr>, PZA00093.2 <chr>,
## #   PZA00096.26 <chr>, PZA00097.13 <chr>, PZA00098.14 <chr>,
## #   PZA00100.10 <chr>, PZA00100.12 <chr>, PZA00100.14 <chr>,
## #   PZA00100.9 <chr>, PZA00103.20 <chr>, PZA00106.9 <chr>,
## #   PZA00107.18 <chr>, PZA00108.12 <chr>, PZA00108.14 <chr>,
## #   PZA00108.15 <chr>, PZA00109.3 <chr>, PZA00109.5 <chr>,
## #   PZA00111.2 <chr>, PZA00111.4 <chr>, PZA00111.5 <chr>,
## #   PZA00111.6 <chr>, PZA00111.8 <chr>, PZA00114.3 <chr>,
## #   PZA00116.2 <chr>, PZA00119.4 <chr>, PZA00120.4 <chr>,
## #   PZA00123.1 <chr>, PZA00125.2 <chr>, PZA00131.14 <chr>,
## #   PZA00132.17 <chr>, PZA00132.18 <chr>, PZA00132.3 <chr>,
## #   PZA00135.6 <chr>, PZA00137.2 <chr>, PZA00139.14 <chr>,
## #   PZA00140.10 <chr>, PZA00140.6 <chr>, PZA00140.9 <chr>,
## #   PZA00142.6 <chr>, PZA00148.2 <chr>, PZA00153.3 <chr>, ...
\end{verbatim}

\begin{Shaded}
\begin{Highlighting}[]
\CommentTok{#head(Geno)}
\KeywordTok{tail}\NormalTok{(Geno)}
\end{Highlighting}
\end{Shaded}

\begin{verbatim}
## # A tibble: 6 x 986
##   Sample_ID JG_OTU   Group abph1.20 abph1.22 ae1.3 ae1.4 ae1.5 an1.4 ba1.6
##   <chr>     <chr>    <chr> <chr>    <chr>    <chr> <chr> <chr> <chr> <chr>
## 1 SYN262    Zmm-IL-~ ZMMIL C/C      A/A      T/T   G/G   C/C   C/C   G/G  
## 2 S0398     Zmm-IL-~ ZMMIL G/G      A/A      T/T   G/G   C/C   C/C   G/G  
## 3 S1636     Zmm-IL-~ ZMMIL G/G      A/A      T/T   G/G   C/C   C/C   G/G  
## 4 CU0201    Zmm-IL-~ ZMMIL C/C      A/A      T/T   G/G   C/C   C/C   G/G  
## 5 S0215     Zmm-IL-~ ZMMIL G/G      A/A      T/T   ?/?   C/C   C/C   G/G  
## 6 CU0202    Zmm-IL-~ ZMMIL C/C      A/A      T/T   G/G   C/C   C/C   ?/?  
## # ... with 976 more variables: ba1.9 <chr>, bt2.5 <chr>, bt2.7 <chr>,
## #   bt2.8 <chr>, Fea2.1 <chr>, Fea2.5 <chr>, id1.3 <chr>, lg2.11 <chr>,
## #   lg2.2 <chr>, pbf1.1 <chr>, pbf1.2 <chr>, pbf1.3 <chr>, pbf1.5 <chr>,
## #   pbf1.6 <chr>, pbf1.7 <chr>, pbf1.8 <chr>, PZA00003.11 <chr>,
## #   PZA00004.2 <chr>, PZA00005.8 <chr>, PZA00005.9 <chr>,
## #   PZA00006.13 <chr>, PZA00006.14 <chr>, PZA00008.1 <chr>,
## #   PZA00010.5 <chr>, PZA00013.10 <chr>, PZA00013.11 <chr>,
## #   PZA00013.9 <chr>, PZA00015.4 <chr>, PZA00017.1 <chr>,
## #   PZA00018.5 <chr>, PZA00029.11 <chr>, PZA00029.12 <chr>,
## #   PZA00030.11 <chr>, PZA00031.5 <chr>, PZA00041.3 <chr>,
## #   PZA00042.2 <chr>, PZA00042.5 <chr>, PZA00043.7 <chr>,
## #   PZA00045.1 <chr>, PZA00047.2 <chr>, PZA00049.12 <chr>,
## #   PZA00050.9 <chr>, PZA00051.2 <chr>, PZA00058.5 <chr>,
## #   PZA00058.6 <chr>, PZA00060.2 <chr>, PZA00061.1 <chr>,
## #   PZA00065.2 <chr>, PZA00069.4 <chr>, PZA00070.5 <chr>,
## #   PZA00078.2 <chr>, PZA00079.1 <chr>, PZA00081.17 <chr>,
## #   PZA00084.2 <chr>, PZA00084.3 <chr>, PZA00086.8 <chr>,
## #   PZA00088.3 <chr>, PZA00090.2 <chr>, PZA00092.1 <chr>,
## #   PZA00092.5 <chr>, PZA00093.2 <chr>, PZA00096.26 <chr>,
## #   PZA00097.13 <chr>, PZA00098.14 <chr>, PZA00100.10 <chr>,
## #   PZA00100.12 <chr>, PZA00100.14 <chr>, PZA00100.9 <chr>,
## #   PZA00103.20 <chr>, PZA00106.9 <chr>, PZA00107.18 <chr>,
## #   PZA00108.12 <chr>, PZA00108.14 <chr>, PZA00108.15 <chr>,
## #   PZA00109.3 <chr>, PZA00109.5 <chr>, PZA00111.2 <chr>,
## #   PZA00111.4 <chr>, PZA00111.5 <chr>, PZA00111.6 <chr>,
## #   PZA00111.8 <chr>, PZA00114.3 <chr>, PZA00116.2 <chr>,
## #   PZA00119.4 <chr>, PZA00120.4 <chr>, PZA00123.1 <chr>,
## #   PZA00125.2 <chr>, PZA00131.14 <chr>, PZA00132.17 <chr>,
## #   PZA00132.18 <chr>, PZA00132.3 <chr>, PZA00135.6 <chr>,
## #   PZA00137.2 <chr>, PZA00139.14 <chr>, PZA00140.10 <chr>,
## #   PZA00140.6 <chr>, PZA00140.9 <chr>, PZA00142.6 <chr>,
## #   PZA00148.2 <chr>, PZA00153.3 <chr>, ...
\end{verbatim}

\begin{Shaded}
\begin{Highlighting}[]
\KeywordTok{dim}\NormalTok{(Geno)}
\end{Highlighting}
\end{Shaded}

\begin{verbatim}
## [1] 2782  986
\end{verbatim}

\begin{Shaded}
\begin{Highlighting}[]
\CommentTok{#str(Geno)}
\KeywordTok{is.list}\NormalTok{(Geno)}
\end{Highlighting}
\end{Shaded}

\begin{verbatim}
## [1] TRUE
\end{verbatim}

\begin{Shaded}
\begin{Highlighting}[]
\KeywordTok{is.matrix}\NormalTok{(Geno)}
\end{Highlighting}
\end{Shaded}

\begin{verbatim}
## [1] FALSE
\end{verbatim}

\begin{Shaded}
\begin{Highlighting}[]
\KeywordTok{is.data.frame}\NormalTok{(Geno)}
\end{Highlighting}
\end{Shaded}

\begin{verbatim}
## [1] TRUE
\end{verbatim}

\begin{Shaded}
\begin{Highlighting}[]
\CommentTok{#colnames(Geno)}

\NormalTok{SNP}
\end{Highlighting}
\end{Shaded}

\begin{verbatim}
## # A tibble: 983 x 15
##    SNP_ID   cdv_marker_id Chromosome Position  alt_pos mult_positions
##    <chr>            <int> <chr>      <chr>     <chr>   <chr>         
##  1 abph1.20          5976 2          27403404  <NA>    <NA>          
##  2 abph1.22          5978 2          27403892  <NA>    <NA>          
##  3 ae1.3             6605 5          167889790 <NA>    <NA>          
##  4 ae1.4             6606 5          167889682 <NA>    <NA>          
##  5 ae1.5             6607 5          167889821 <NA>    <NA>          
##  6 an1.4             5982 1          240498509 <NA>    <NA>          
##  7 ba1.6             3463 3          181362952 <NA>    <NA>          
##  8 ba1.9             3466 3          181362666 <NA>    <NA>          
##  9 bt2.5             5983 4          66290049  <NA>    <NA>          
## 10 bt2.7             5985 4          66290994  <NA>    <NA>          
## # ... with 973 more rows, and 9 more variables: amplicon <chr>,
## #   cdv_map_feature.name <chr>, gene <chr>, `candidate/random` <chr>,
## #   Genaissance_daa_id <int>, Sequenom_daa_id <int>,
## #   count_amplicons <int>, count_cmf <int>, count_gene <int>
\end{verbatim}

\begin{Shaded}
\begin{Highlighting}[]
\CommentTok{#view(SNP)}
\CommentTok{#head(SNP)}
\KeywordTok{tail}\NormalTok{(SNP)}
\end{Highlighting}
\end{Shaded}

\begin{verbatim}
## # A tibble: 6 x 15
##   SNP_ID cdv_marker_id Chromosome Position alt_pos mult_positions amplicon
##   <chr>          <int> <chr>      <chr>    <chr>   <chr>          <chr>   
## 1 zap1.2          3514 2          2331285~ <NA>    <NA>           zap1    
## 2 zen1.1          3519 unknown    unknown  <NA>    <NA>           zen1    
## 3 zen1.2          3520 unknown    unknown  <NA>    <NA>           zen1    
## 4 zen1.4          3521 unknown    unknown  <NA>    <NA>           zen1    
## 5 zfl2.6          6463 2          12543294 <NA>    <NA>           zfl2    
## 6 zmm3.4          3527 9          16966348 <NA>    <NA>           zmm3    
## # ... with 8 more variables: cdv_map_feature.name <chr>, gene <chr>,
## #   `candidate/random` <chr>, Genaissance_daa_id <int>,
## #   Sequenom_daa_id <int>, count_amplicons <int>, count_cmf <int>,
## #   count_gene <int>
\end{verbatim}

\begin{Shaded}
\begin{Highlighting}[]
\KeywordTok{str}\NormalTok{(SNP)}
\end{Highlighting}
\end{Shaded}

\begin{verbatim}
## Classes 'tbl_df', 'tbl' and 'data.frame':    983 obs. of  15 variables:
##  $ SNP_ID              : chr  "abph1.20" "abph1.22" "ae1.3" "ae1.4" ...
##  $ cdv_marker_id       : int  5976 5978 6605 6606 6607 5982 3463 3466 5983 5985 ...
##  $ Chromosome          : chr  "2" "2" "5" "5" ...
##  $ Position            : chr  "27403404" "27403892" "167889790" "167889682" ...
##  $ alt_pos             : chr  NA NA NA NA ...
##  $ mult_positions      : chr  NA NA NA NA ...
##  $ amplicon            : chr  "abph1" "abph1" "ae1" "ae1" ...
##  $ cdv_map_feature.name: chr  "AB042260" "AB042260" "ae1" "ae1" ...
##  $ gene                : chr  "abph1" "abph1" "ae1" "ae1" ...
##  $ candidate/random    : chr  "candidate" "candidate" "candidate" "candidate" ...
##  $ Genaissance_daa_id  : int  8393 8394 8395 8396 8397 8398 8399 8400 8401 8402 ...
##  $ Sequenom_daa_id     : int  10474 10475 10477 10478 10479 10481 10482 10483 10486 10487 ...
##  $ count_amplicons     : int  1 0 1 0 0 1 1 0 1 0 ...
##  $ count_cmf           : int  1 0 1 0 0 1 0 0 1 0 ...
##  $ count_gene          : int  1 0 1 0 0 1 1 0 1 0 ...
##  - attr(*, "spec")=List of 2
##   ..$ cols   :List of 15
##   .. ..$ SNP_ID              : list()
##   .. .. ..- attr(*, "class")= chr  "collector_character" "collector"
##   .. ..$ cdv_marker_id       : list()
##   .. .. ..- attr(*, "class")= chr  "collector_integer" "collector"
##   .. ..$ Chromosome          : list()
##   .. .. ..- attr(*, "class")= chr  "collector_character" "collector"
##   .. ..$ Position            : list()
##   .. .. ..- attr(*, "class")= chr  "collector_character" "collector"
##   .. ..$ alt_pos             : list()
##   .. .. ..- attr(*, "class")= chr  "collector_character" "collector"
##   .. ..$ mult_positions      : list()
##   .. .. ..- attr(*, "class")= chr  "collector_character" "collector"
##   .. ..$ amplicon            : list()
##   .. .. ..- attr(*, "class")= chr  "collector_character" "collector"
##   .. ..$ cdv_map_feature.name: list()
##   .. .. ..- attr(*, "class")= chr  "collector_character" "collector"
##   .. ..$ gene                : list()
##   .. .. ..- attr(*, "class")= chr  "collector_character" "collector"
##   .. ..$ candidate/random    : list()
##   .. .. ..- attr(*, "class")= chr  "collector_character" "collector"
##   .. ..$ Genaissance_daa_id  : list()
##   .. .. ..- attr(*, "class")= chr  "collector_integer" "collector"
##   .. ..$ Sequenom_daa_id     : list()
##   .. .. ..- attr(*, "class")= chr  "collector_integer" "collector"
##   .. ..$ count_amplicons     : list()
##   .. .. ..- attr(*, "class")= chr  "collector_integer" "collector"
##   .. ..$ count_cmf           : list()
##   .. .. ..- attr(*, "class")= chr  "collector_integer" "collector"
##   .. ..$ count_gene          : list()
##   .. .. ..- attr(*, "class")= chr  "collector_integer" "collector"
##   ..$ default: list()
##   .. ..- attr(*, "class")= chr  "collector_guess" "collector"
##   ..- attr(*, "class")= chr "col_spec"
\end{verbatim}

\begin{Shaded}
\begin{Highlighting}[]
\KeywordTok{is.list}\NormalTok{(SNP)}
\end{Highlighting}
\end{Shaded}

\begin{verbatim}
## [1] TRUE
\end{verbatim}

\begin{Shaded}
\begin{Highlighting}[]
\KeywordTok{is.matrix}\NormalTok{(SNP)}
\end{Highlighting}
\end{Shaded}

\begin{verbatim}
## [1] FALSE
\end{verbatim}

\begin{Shaded}
\begin{Highlighting}[]
\KeywordTok{is.data.frame}\NormalTok{(SNP)}
\end{Highlighting}
\end{Shaded}

\begin{verbatim}
## [1] TRUE
\end{verbatim}

\begin{Shaded}
\begin{Highlighting}[]
\KeywordTok{dim}\NormalTok{(SNP)}
\end{Highlighting}
\end{Shaded}

\begin{verbatim}
## [1] 983  15
\end{verbatim}

\begin{Shaded}
\begin{Highlighting}[]
\CommentTok{#attributes(SNP)}
\KeywordTok{summary}\NormalTok{(SNP)}
\end{Highlighting}
\end{Shaded}

\begin{verbatim}
##     SNP_ID          cdv_marker_id    Chromosome          Position        
##  Length:983         Min.   : 3463   Length:983         Length:983        
##  Class :character   1st Qu.: 3978   Class :character   Class :character  
##  Mode  :character   Median : 5723   Mode  :character   Mode  :character  
##                     Mean   : 5925                                        
##                     3rd Qu.: 6629                                        
##                     Max.   :12480                                        
##    alt_pos          mult_positions       amplicon        
##  Length:983         Length:983         Length:983        
##  Class :character   Class :character   Class :character  
##  Mode  :character   Mode  :character   Mode  :character  
##                                                          
##                                                          
##                                                          
##  cdv_map_feature.name     gene           candidate/random  
##  Length:983           Length:983         Length:983        
##  Class :character     Class :character   Class :character  
##  Mode  :character     Mode  :character   Mode  :character  
##                                                            
##                                                            
##                                                            
##  Genaissance_daa_id Sequenom_daa_id count_amplicons    count_cmf     
##  Min.   : 7649      Min.   :10474   Min.   :0.0000   Min.   :0.0000  
##  1st Qu.: 7906      1st Qu.:10784   1st Qu.:0.0000   1st Qu.:0.0000  
##  Median : 8173      Median :11110   Median :1.0000   Median :1.0000  
##  Mean   : 8524      Mean   :11122   Mean   :0.5768   Mean   :0.5483  
##  3rd Qu.: 9834      3rd Qu.:11420   3rd Qu.:1.0000   3rd Qu.:1.0000  
##  Max.   :10104      Max.   :11829   Max.   :1.0000   Max.   :1.0000  
##    count_gene    
##  Min.   :0.0000  
##  1st Qu.:0.0000  
##  Median :1.0000  
##  Mean   :0.5565  
##  3rd Qu.:1.0000  
##  Max.   :1.0000
\end{verbatim}

\begin{Shaded}
\begin{Highlighting}[]
\KeywordTok{colnames}\NormalTok{(SNP)}
\end{Highlighting}
\end{Shaded}

\begin{verbatim}
##  [1] "SNP_ID"               "cdv_marker_id"        "Chromosome"          
##  [4] "Position"             "alt_pos"              "mult_positions"      
##  [7] "amplicon"             "cdv_map_feature.name" "gene"                
## [10] "candidate/random"     "Genaissance_daa_id"   "Sequenom_daa_id"     
## [13] "count_amplicons"      "count_cmf"            "count_gene"
\end{verbatim}

\subsection{SNP Data trimming and subset Geno file for Maize and
Teosinte
files}\label{snp-data-trimming-and-subset-geno-file-for-maize-and-teosinte-files}

\begin{Shaded}
\begin{Highlighting}[]
\CommentTok{#Creat reduced SNP file with only SNP_ID [1], Chromosome [3], and Position [4]}
\NormalTok{RedSNP <-}\StringTok{ }\NormalTok{SNP[}\KeywordTok{c}\NormalTok{(}\DecValTok{1}\NormalTok{,}\DecValTok{3}\NormalTok{,}\DecValTok{4}\NormalTok{)] }\CommentTok{# or use select(SNP, SNP_ID, Chromosome, Position)}

\CommentTok{#subset the Geno file into Maize and Teosinte}
\NormalTok{Maize <-}\StringTok{ }\KeywordTok{filter}\NormalTok{(Geno, Group }\OperatorTok{==}\StringTok{'ZMMIL'}\OperatorTok{|}\NormalTok{Group}\OperatorTok{==}\StringTok{'ZMMLR'}\OperatorTok{|}\NormalTok{Group}\OperatorTok{==}\StringTok{'ZMMMR'}\NormalTok{)}
\NormalTok{Maize <-}\StringTok{ }\NormalTok{Maize[,}\KeywordTok{c}\NormalTok{(}\OperatorTok{-}\DecValTok{2}\NormalTok{,}\OperatorTok{-}\DecValTok{3}\NormalTok{)] }\CommentTok{# To remove unused cols. Or use select(Maize,-(JG_OTU, Group))}
\NormalTok{Maize}
\end{Highlighting}
\end{Shaded}

\begin{verbatim}
## # A tibble: 1,573 x 984
##    Sample_ID abph1.20 abph1.22 ae1.3 ae1.4 ae1.5 an1.4 ba1.6 ba1.9 bt2.5
##    <chr>     <chr>    <chr>    <chr> <chr> <chr> <chr> <chr> <chr> <chr>
##  1 ZDP_0752a C/G      A/A      T/T   G/G   C/C   C/G   G/G   G/G   C/C  
##  2 ZDP_0793a C/G      A/A      T/T   G/G   C/T   C/G   A/G   G/T   C/C  
##  3 ZDP_0612a C/C      A/A      T/T   G/G   C/C   C/C   G/G   ?/?   C/C  
##  4 ZDP_0602a C/G      A/A      G/T   A/G   C/T   C/C   G/G   G/G   C/C  
##  5 ZDP_0581a C/C      A/A      T/T   G/G   C/T   C/C   G/G   G/G   C/C  
##  6 ZDP_0552a C/G      A/A      T/T   G/G   C/T   C/C   G/G   G/G   C/C  
##  7 ZDP_0543a C/C      A/A      T/T   G/G   C/C   C/C   G/G   G/G   C/C  
##  8 ZDP_0042a C/C      A/A      T/T   G/G   C/T   C/C   G/G   G/G   C/C  
##  9 ZDP_0121a C/C      A/A      T/T   ?/?   C/T   C/C   G/G   G/G   C/C  
## 10 ZDP_0101b C/C      A/A      G/T   A/G   T/T   C/C   G/G   G/G   C/C  
## # ... with 1,563 more rows, and 974 more variables: bt2.7 <chr>,
## #   bt2.8 <chr>, Fea2.1 <chr>, Fea2.5 <chr>, id1.3 <chr>, lg2.11 <chr>,
## #   lg2.2 <chr>, pbf1.1 <chr>, pbf1.2 <chr>, pbf1.3 <chr>, pbf1.5 <chr>,
## #   pbf1.6 <chr>, pbf1.7 <chr>, pbf1.8 <chr>, PZA00003.11 <chr>,
## #   PZA00004.2 <chr>, PZA00005.8 <chr>, PZA00005.9 <chr>,
## #   PZA00006.13 <chr>, PZA00006.14 <chr>, PZA00008.1 <chr>,
## #   PZA00010.5 <chr>, PZA00013.10 <chr>, PZA00013.11 <chr>,
## #   PZA00013.9 <chr>, PZA00015.4 <chr>, PZA00017.1 <chr>,
## #   PZA00018.5 <chr>, PZA00029.11 <chr>, PZA00029.12 <chr>,
## #   PZA00030.11 <chr>, PZA00031.5 <chr>, PZA00041.3 <chr>,
## #   PZA00042.2 <chr>, PZA00042.5 <chr>, PZA00043.7 <chr>,
## #   PZA00045.1 <chr>, PZA00047.2 <chr>, PZA00049.12 <chr>,
## #   PZA00050.9 <chr>, PZA00051.2 <chr>, PZA00058.5 <chr>,
## #   PZA00058.6 <chr>, PZA00060.2 <chr>, PZA00061.1 <chr>,
## #   PZA00065.2 <chr>, PZA00069.4 <chr>, PZA00070.5 <chr>,
## #   PZA00078.2 <chr>, PZA00079.1 <chr>, PZA00081.17 <chr>,
## #   PZA00084.2 <chr>, PZA00084.3 <chr>, PZA00086.8 <chr>,
## #   PZA00088.3 <chr>, PZA00090.2 <chr>, PZA00092.1 <chr>,
## #   PZA00092.5 <chr>, PZA00093.2 <chr>, PZA00096.26 <chr>,
## #   PZA00097.13 <chr>, PZA00098.14 <chr>, PZA00100.10 <chr>,
## #   PZA00100.12 <chr>, PZA00100.14 <chr>, PZA00100.9 <chr>,
## #   PZA00103.20 <chr>, PZA00106.9 <chr>, PZA00107.18 <chr>,
## #   PZA00108.12 <chr>, PZA00108.14 <chr>, PZA00108.15 <chr>,
## #   PZA00109.3 <chr>, PZA00109.5 <chr>, PZA00111.2 <chr>,
## #   PZA00111.4 <chr>, PZA00111.5 <chr>, PZA00111.6 <chr>,
## #   PZA00111.8 <chr>, PZA00114.3 <chr>, PZA00116.2 <chr>,
## #   PZA00119.4 <chr>, PZA00120.4 <chr>, PZA00123.1 <chr>,
## #   PZA00125.2 <chr>, PZA00131.14 <chr>, PZA00132.17 <chr>,
## #   PZA00132.18 <chr>, PZA00132.3 <chr>, PZA00135.6 <chr>,
## #   PZA00137.2 <chr>, PZA00139.14 <chr>, PZA00140.10 <chr>,
## #   PZA00140.6 <chr>, PZA00140.9 <chr>, PZA00142.6 <chr>,
## #   PZA00148.2 <chr>, PZA00153.3 <chr>, PZA00153.6 <chr>,
## #   PZA00163.4 <chr>, ...
\end{verbatim}

\begin{Shaded}
\begin{Highlighting}[]
\NormalTok{Teosinte <-}\StringTok{ }\KeywordTok{filter}\NormalTok{(Geno, Group }\OperatorTok{==}\StringTok{ 'ZMPBA'}\OperatorTok{|}\StringTok{ }\NormalTok{Group}\OperatorTok{==}\StringTok{'ZMPIL'}\OperatorTok{|}\StringTok{ }\NormalTok{Group}\OperatorTok{==}\StringTok{'ZMPJA'}\NormalTok{)}
\NormalTok{Teosinte <-}\StringTok{ }\NormalTok{Teosinte[,}\KeywordTok{c}\NormalTok{(}\OperatorTok{-}\DecValTok{2}\NormalTok{,}\OperatorTok{-}\DecValTok{3}\NormalTok{)] }\CommentTok{# To remove unused cols}
\NormalTok{Teosinte}
\end{Highlighting}
\end{Shaded}

\begin{verbatim}
## # A tibble: 975 x 984
##    Sample_ID abph1.20 abph1.22 ae1.3 ae1.4 ae1.5 an1.4 ba1.6 ba1.9 bt2.5
##    <chr>     <chr>    <chr>    <chr> <chr> <chr> <chr> <chr> <chr> <chr>
##  1 S0881     C/G      A/A      T/T   G/G   T/T   C/C   A/G   G/G   T/T  
##  2 S1057     C/G      A/A      G/T   A/G   T/T   C/C   A/A   G/G   C/T  
##  3 S1087     G/G      A/A      T/T   G/G   T/T   C/C   A/A   G/G   C/T  
##  4 S1689     G/G      A/A      T/T   G/G   T/T   C/C   A/A   G/G   C/C  
##  5 S1697     C/G      A/A      T/T   G/G   T/T   C/C   A/A   G/G   C/C  
##  6 S1698     C/G      A/A      T/T   A/G   T/T   C/C   A/A   G/G   C/T  
##  7 S1703     G/G      A/A      T/T   G/G   T/T   C/C   A/A   G/G   C/C  
##  8 S1704     G/G      A/A      T/T   G/G   C/T   C/C   A/A   T/T   C/T  
##  9 TA057     G/G      A/A      T/T   G/G   C/T   C/C   A/A   G/T   C/C  
## 10 TA137     G/G      A/A      T/T   G/G   T/T   C/C   A/A   G/T   C/T  
## # ... with 965 more rows, and 974 more variables: bt2.7 <chr>,
## #   bt2.8 <chr>, Fea2.1 <chr>, Fea2.5 <chr>, id1.3 <chr>, lg2.11 <chr>,
## #   lg2.2 <chr>, pbf1.1 <chr>, pbf1.2 <chr>, pbf1.3 <chr>, pbf1.5 <chr>,
## #   pbf1.6 <chr>, pbf1.7 <chr>, pbf1.8 <chr>, PZA00003.11 <chr>,
## #   PZA00004.2 <chr>, PZA00005.8 <chr>, PZA00005.9 <chr>,
## #   PZA00006.13 <chr>, PZA00006.14 <chr>, PZA00008.1 <chr>,
## #   PZA00010.5 <chr>, PZA00013.10 <chr>, PZA00013.11 <chr>,
## #   PZA00013.9 <chr>, PZA00015.4 <chr>, PZA00017.1 <chr>,
## #   PZA00018.5 <chr>, PZA00029.11 <chr>, PZA00029.12 <chr>,
## #   PZA00030.11 <chr>, PZA00031.5 <chr>, PZA00041.3 <chr>,
## #   PZA00042.2 <chr>, PZA00042.5 <chr>, PZA00043.7 <chr>,
## #   PZA00045.1 <chr>, PZA00047.2 <chr>, PZA00049.12 <chr>,
## #   PZA00050.9 <chr>, PZA00051.2 <chr>, PZA00058.5 <chr>,
## #   PZA00058.6 <chr>, PZA00060.2 <chr>, PZA00061.1 <chr>,
## #   PZA00065.2 <chr>, PZA00069.4 <chr>, PZA00070.5 <chr>,
## #   PZA00078.2 <chr>, PZA00079.1 <chr>, PZA00081.17 <chr>,
## #   PZA00084.2 <chr>, PZA00084.3 <chr>, PZA00086.8 <chr>,
## #   PZA00088.3 <chr>, PZA00090.2 <chr>, PZA00092.1 <chr>,
## #   PZA00092.5 <chr>, PZA00093.2 <chr>, PZA00096.26 <chr>,
## #   PZA00097.13 <chr>, PZA00098.14 <chr>, PZA00100.10 <chr>,
## #   PZA00100.12 <chr>, PZA00100.14 <chr>, PZA00100.9 <chr>,
## #   PZA00103.20 <chr>, PZA00106.9 <chr>, PZA00107.18 <chr>,
## #   PZA00108.12 <chr>, PZA00108.14 <chr>, PZA00108.15 <chr>,
## #   PZA00109.3 <chr>, PZA00109.5 <chr>, PZA00111.2 <chr>,
## #   PZA00111.4 <chr>, PZA00111.5 <chr>, PZA00111.6 <chr>,
## #   PZA00111.8 <chr>, PZA00114.3 <chr>, PZA00116.2 <chr>,
## #   PZA00119.4 <chr>, PZA00120.4 <chr>, PZA00123.1 <chr>,
## #   PZA00125.2 <chr>, PZA00131.14 <chr>, PZA00132.17 <chr>,
## #   PZA00132.18 <chr>, PZA00132.3 <chr>, PZA00135.6 <chr>,
## #   PZA00137.2 <chr>, PZA00139.14 <chr>, PZA00140.10 <chr>,
## #   PZA00140.6 <chr>, PZA00140.9 <chr>, PZA00142.6 <chr>,
## #   PZA00148.2 <chr>, PZA00153.3 <chr>, PZA00153.6 <chr>,
## #   PZA00163.4 <chr>, ...
\end{verbatim}

\subsection{Transpose}\label{transpose}

Transposed files will lose the header, so I took the colnames from the
orginal Maize file (vector) and combine with the transposed file using
cbind (column bind), to add in the first column. cbind tends to convert
character columns to fators unleass \texttt{stringAsFactors\ =\ F} is
applied. as.tibble was used to easlily view the dataframe that has been
transposed and cbinded. After transposition, there is neither the row
names, nor the column names specified. So the first row was used to set
as the colnames for tMaize. A duplicate row was removed with
\texttt{tMaize\ =\ tMaize{[}-1,\ {]}}. In order to joining files later,
the first cell was replaced from ``Sample\_ID'' to ``SNP\_ID''.

\begin{Shaded}
\begin{Highlighting}[]
\NormalTok{tMaize<-}\StringTok{ }\KeywordTok{t}\NormalTok{(Maize)}
\NormalTok{tMaize<-}\KeywordTok{cbind}\NormalTok{(}\KeywordTok{colnames}\NormalTok{(Maize),tMaize,}\DataTypeTok{stringsAsFactors =}\NormalTok{ F)}
\NormalTok{tMaize<-}\KeywordTok{as.tibble}\NormalTok{(tMaize)}
\KeywordTok{colnames}\NormalTok{(tMaize)<-}\StringTok{ }\NormalTok{tMaize[}\DecValTok{1}\NormalTok{,]}
\NormalTok{tMaizee =}\StringTok{ }\NormalTok{tMaize[}\OperatorTok{-}\DecValTok{1}\NormalTok{, ]}
\KeywordTok{colnames}\NormalTok{(tMaize)[}\DecValTok{1}\NormalTok{] <-}\StringTok{ "SNP_ID"}
\NormalTok{tMaize}
\end{Highlighting}
\end{Shaded}

\begin{verbatim}
## # A tibble: 984 x 1,575
##    SNP_ID    ZDP_0752a ZDP_0793a ZDP_0612a ZDP_0602a ZDP_0581a ZDP_0552a
##    <chr>     <chr>     <chr>     <chr>     <chr>     <chr>     <chr>    
##  1 Sample_ID ZDP_0752a ZDP_0793a ZDP_0612a ZDP_0602a ZDP_0581a ZDP_0552a
##  2 abph1.20  C/G       C/G       C/C       C/G       C/C       C/G      
##  3 abph1.22  A/A       A/A       A/A       A/A       A/A       A/A      
##  4 ae1.3     T/T       T/T       T/T       G/T       T/T       T/T      
##  5 ae1.4     G/G       G/G       G/G       A/G       G/G       G/G      
##  6 ae1.5     C/C       C/T       C/C       C/T       C/T       C/T      
##  7 an1.4     C/G       C/G       C/C       C/C       C/C       C/C      
##  8 ba1.6     G/G       A/G       G/G       G/G       G/G       G/G      
##  9 ba1.9     G/G       G/T       ?/?       G/G       G/G       G/G      
## 10 bt2.5     C/C       C/C       C/C       C/C       C/C       C/C      
## # ... with 974 more rows, and 1,568 more variables: ZDP_0543a <chr>,
## #   ZDP_0042a <chr>, ZDP_0121a <chr>, ZDP_0101b <chr>, ZDP_0087a <chr>,
## #   ZDP_0033a <chr>, ZDP_0373b <chr>, ZDP_0168a <chr>, ZDP_0169a <chr>,
## #   ZDP_0517b <chr>, ZDP_1098a <chr>, ZDP_0591a <chr>, ZDP_0258a <chr>,
## #   ZDP_0009a <chr>, ZDP_0016a <chr>, ZDP_0010a <chr>, ZDP_0018a <chr>,
## #   ZDP_0647b <chr>, ZDP_1064b <chr>, ZDP_1053a <chr>, ZDP_1445b <chr>,
## #   ZDP_0163a <chr>, ZDP_0164a <chr>, ZDP_0166a <chr>, ZDP_0839a <chr>,
## #   ZDP_0849a <chr>, ZDP_0858b <chr>, ZDP_0184b <chr>, ZDP_0179a <chr>,
## #   ZDP_0180a <chr>, ZDP_1021a <chr>, ZDP_0181a <chr>, ZDP_1004a <chr>,
## #   S1799 <chr>, ZDP_0856a <chr>, ZDP_0915b <chr>, ZDP_1194b <chr>,
## #   ZDP_1476a <chr>, ZDP_0935a <chr>, ZDP_0197a <chr>, ZDP_0206a <chr>,
## #   ZDP_0195a <chr>, ZDP_0051a <chr>, ZDP_0054a <chr>, ZDP_0052a <chr>,
## #   ZDP_0675b <chr>, ZDP_0791a <chr>, ZDP_0814a <chr>, ZDP_0122a <chr>,
## #   ZDP_0090a <chr>, ZDP_0083a <chr>, ZDP_0084a <chr>, ZDP_0061a <chr>,
## #   ZDP_0561a <chr>, ZDP_0057a <chr>, ZDP_0570a <chr>, ZDP_0899b <chr>,
## #   ZDP_0888b <chr>, ZDP_1104a <chr>, ZDP_1162a <chr>, S1579 <chr>,
## #   ZDP_0634a <chr>, S1843 <chr>, ZDP_0689b <chr>, ZDP_0733b <chr>,
## #   ZDP_0758b <chr>, ZDP_1003b <chr>, ZDP_1012a <chr>, ZDP_0640b <chr>,
## #   ZDP_0153a <chr>, ZDP_0154a <chr>, ZDP_1020a <chr>, S1822 <chr>,
## #   ZDP_0994a <chr>, ZDP_1208a <chr>, ZDP_1489a <chr>, ZDP_0123b <chr>,
## #   ZDP_1010a <chr>, ZDP_0001a <chr>, ZDP_0174a <chr>, ZDP_0178a <chr>,
## #   ZDP_1221b <chr>, ZDP_1481a <chr>, ZDP_0988a <chr>, ZDP_0456a <chr>,
## #   ZDP_0368b <chr>, ZDP_0376a <chr>, BKN_003 <chr>, BKN_004 <chr>,
## #   ZDP_0396a <chr>, ZDP_0417a <chr>, ZDP_0563a <chr>, ZDP_0727a <chr>,
## #   ZDP_0746b <chr>, ZDP_0756b <chr>, ZDP_1126a <chr>, ZDP_1186a <chr>,
## #   ZDP_1100a <chr>, ZDP_0496a <chr>, ZDP_0280a <chr>, ...
\end{verbatim}

\begin{Shaded}
\begin{Highlighting}[]
\NormalTok{tTeosinte<-}\StringTok{ }\KeywordTok{t}\NormalTok{(Teosinte)}
\NormalTok{tTeosinte<-}\KeywordTok{cbind}\NormalTok{(}\KeywordTok{colnames}\NormalTok{(Teosinte),tTeosinte,}\DataTypeTok{stringsAsFactors =}\NormalTok{ F)}
\NormalTok{tTeosinte<-}\KeywordTok{as.tibble}\NormalTok{(tTeosinte)}
\KeywordTok{colnames}\NormalTok{(tTeosinte)<-}\StringTok{ }\NormalTok{tTeosinte[}\DecValTok{1}\NormalTok{,]}
\NormalTok{tTeosinte =}\StringTok{ }\NormalTok{tTeosinte[}\OperatorTok{-}\DecValTok{1}\NormalTok{, ]}
\KeywordTok{colnames}\NormalTok{(tTeosinte)[}\DecValTok{1}\NormalTok{] <-}\StringTok{ "SNP_ID"}
\NormalTok{tTeosinte}
\end{Highlighting}
\end{Shaded}

\begin{verbatim}
## # A tibble: 983 x 977
##    SNP_ID   S0881 S1057 S1087 S1689 S1697 S1698 S1703 S1704 TA057 TA137
##    <chr>    <chr> <chr> <chr> <chr> <chr> <chr> <chr> <chr> <chr> <chr>
##  1 abph1.20 C/G   C/G   G/G   G/G   C/G   C/G   G/G   G/G   G/G   G/G  
##  2 abph1.22 A/A   A/A   A/A   A/A   A/A   A/A   A/A   A/A   A/A   A/A  
##  3 ae1.3    T/T   G/T   T/T   T/T   T/T   T/T   T/T   T/T   T/T   T/T  
##  4 ae1.4    G/G   A/G   G/G   G/G   G/G   A/G   G/G   G/G   G/G   G/G  
##  5 ae1.5    T/T   T/T   T/T   T/T   T/T   T/T   T/T   C/T   C/T   T/T  
##  6 an1.4    C/C   C/C   C/C   C/C   C/C   C/C   C/C   C/C   C/C   C/C  
##  7 ba1.6    A/G   A/A   A/A   A/A   A/A   A/A   A/A   A/A   A/A   A/A  
##  8 ba1.9    G/G   G/G   G/G   G/G   G/G   G/G   G/G   T/T   G/T   G/T  
##  9 bt2.5    T/T   C/T   C/T   C/C   C/C   C/T   C/C   C/T   C/C   C/T  
## 10 bt2.7    A/A   A/A   A/A   A/A   A/A   A/A   A/A   A/A   A/A   A/A  
## # ... with 973 more rows, and 966 more variables: TA153 <chr>,
## #   TA205 <chr>, TA217 <chr>, TA242 <chr>, TA285 <chr>, TA293 <chr>,
## #   TAMex0091.1 <chr>, TAMex0153.1 <chr>, TAMex0238.1 <chr>,
## #   TAMex0396.1 <chr>, TAMex0446.1 <chr>, TAMex0455.2 <chr>,
## #   TAMex0520.1 <chr>, TAMex0608.5 <chr>, TAMex0624.1 <chr>,
## #   TAMex0643.1 <chr>, TAMex0775.2 <chr>, TAMex0930.1 <chr>,
## #   TAMex1083.1 <chr>, TAMex1111.1 <chr>, TAMex1195.2 <chr>,
## #   TAMex1222.2 <chr>, TAMex1308.1 <chr>, TAMex1321.1 <chr>,
## #   TAMex1344.2 <chr>, TAMex1422.2 <chr>, TAMex1488.1 <chr>,
## #   TAMex1561.2 <chr>, TAMex0005.1 <chr>, TAMex0077.1 <chr>,
## #   TAMex0240.2 <chr>, TAMex0299.1 <chr>, TAMex0382.2 <chr>,
## #   TAMex0482.2 <chr>, TAMex0483.1 <chr>, TAMex0488.1 <chr>,
## #   TAMex0509.1 <chr>, TAMex0621.2 <chr>, TAMex0819.1 <chr>,
## #   TAMex0840.1 <chr>, TAMex0855.2 <chr>, TAMex0901.2 <chr>,
## #   TAMex0925.1 <chr>, TAMex0942.1 <chr>, TAMex0966.1 <chr>,
## #   TAMex1005.1 <chr>, TAMex1136.1 <chr>, TAMex1165.1 <chr>,
## #   TAMex1340.1 <chr>, TAMex1349.2 <chr>, TAMex1391.2 <chr>,
## #   TAMex1437.1 <chr>, TAMex1508.4 <chr>, TAMex0050.1 <chr>,
## #   TAMex0106.1 <chr>, TAMex0167.1 <chr>, TAMex0181.2 <chr>,
## #   TAMex0207.2 <chr>, TAMex0438.1 <chr>, TAMex0515.1 <chr>,
## #   TAMex0517.1 <chr>, TAMex0568.1 <chr>, TAMex0645.1 <chr>,
## #   TAMex0667.1 <chr>, TAMex0757.1 <chr>, TAMex0845.2 <chr>,
## #   TAMex0849.3 <chr>, TAMex0918.2 <chr>, TAMex0947.1 <chr>,
## #   TAMex0982.1 <chr>, TAMex1139.2 <chr>, TAMex1298.4 <chr>,
## #   TAMex1348.3 <chr>, TAMex1358.1 <chr>, TAMex1439.2 <chr>,
## #   TAMex1443.1 <chr>, TAMex1467.1 <chr>, TAMex0267.2 <chr>,
## #   TAMex0280.3 <chr>, TAMex0400.1 <chr>, TAMex0403.1 <chr>,
## #   TAMex0412.1 <chr>, TAMex0495.1 <chr>, TAMex0514.1 <chr>,
## #   TAMex0544.1 <chr>, TAMex0587.1 <chr>, TAMex0613.2 <chr>,
## #   TAMex0766.2 <chr>, TAMex0785.1 <chr>, TAMex0874.1 <chr>,
## #   TAMex0897.2 <chr>, TAMex1069.2 <chr>, TAMex1087.1 <chr>,
## #   TAMex1107.1 <chr>, TAMex1160.1 <chr>, TAMex1178.1 <chr>,
## #   TAMex1246.3 <chr>, TAMex1271.2 <chr>, TAMex1283.1 <chr>,
## #   TAMex1429.2 <chr>, ...
\end{verbatim}

\subsection{Merge SNP file with Maize and Teosinte
files}\label{merge-snp-file-with-maize-and-teosinte-files}

I used \texttt{merge} function here, which is base R function. However,
\texttt{inner\_join} function from dplyr can be applied, so that
\texttt{as.tibble} can be dismissed.
\texttt{SNP\_tMaize1\textless{}-\ inner\_join(RedSNP,tMaize,\ by=\ "SNP\_ID")}

\begin{Shaded}
\begin{Highlighting}[]
\NormalTok{SNP_tMaize<-}\StringTok{ }\KeywordTok{merge}\NormalTok{ (RedSNP,tMaize, }\DataTypeTok{by=} \StringTok{"SNP_ID"}\NormalTok{) }\CommentTok{#method #1}
\NormalTok{SNP_tMaize<-}\KeywordTok{as.tibble}\NormalTok{(SNP_tMaize)}
\NormalTok{SNP_tMaize}
\end{Highlighting}
\end{Shaded}

\begin{verbatim}
## # A tibble: 983 x 1,577
##    SNP_ID   Chromosome Position  ZDP_0752a ZDP_0793a ZDP_0612a ZDP_0602a
##    <chr>    <chr>      <chr>     <chr>     <chr>     <chr>     <chr>    
##  1 abph1.20 2          27403404  C/G       C/G       C/C       C/G      
##  2 abph1.22 2          27403892  A/A       A/A       A/A       A/A      
##  3 ae1.3    5          167889790 T/T       T/T       T/T       G/T      
##  4 ae1.4    5          167889682 G/G       G/G       G/G       A/G      
##  5 ae1.5    5          167889821 C/C       C/T       C/C       C/T      
##  6 an1.4    1          240498509 C/G       C/G       C/C       C/C      
##  7 ba1.6    3          181362952 G/G       A/G       G/G       G/G      
##  8 ba1.9    3          181362666 G/G       G/T       ?/?       G/G      
##  9 bt2.5    4          66290049  C/C       C/C       C/C       C/C      
## 10 bt2.7    4          66290994  ?/?       ?/?       G/G       G/G      
## # ... with 973 more rows, and 1,570 more variables: ZDP_0581a <chr>,
## #   ZDP_0552a <chr>, ZDP_0543a <chr>, ZDP_0042a <chr>, ZDP_0121a <chr>,
## #   ZDP_0101b <chr>, ZDP_0087a <chr>, ZDP_0033a <chr>, ZDP_0373b <chr>,
## #   ZDP_0168a <chr>, ZDP_0169a <chr>, ZDP_0517b <chr>, ZDP_1098a <chr>,
## #   ZDP_0591a <chr>, ZDP_0258a <chr>, ZDP_0009a <chr>, ZDP_0016a <chr>,
## #   ZDP_0010a <chr>, ZDP_0018a <chr>, ZDP_0647b <chr>, ZDP_1064b <chr>,
## #   ZDP_1053a <chr>, ZDP_1445b <chr>, ZDP_0163a <chr>, ZDP_0164a <chr>,
## #   ZDP_0166a <chr>, ZDP_0839a <chr>, ZDP_0849a <chr>, ZDP_0858b <chr>,
## #   ZDP_0184b <chr>, ZDP_0179a <chr>, ZDP_0180a <chr>, ZDP_1021a <chr>,
## #   ZDP_0181a <chr>, ZDP_1004a <chr>, S1799 <chr>, ZDP_0856a <chr>,
## #   ZDP_0915b <chr>, ZDP_1194b <chr>, ZDP_1476a <chr>, ZDP_0935a <chr>,
## #   ZDP_0197a <chr>, ZDP_0206a <chr>, ZDP_0195a <chr>, ZDP_0051a <chr>,
## #   ZDP_0054a <chr>, ZDP_0052a <chr>, ZDP_0675b <chr>, ZDP_0791a <chr>,
## #   ZDP_0814a <chr>, ZDP_0122a <chr>, ZDP_0090a <chr>, ZDP_0083a <chr>,
## #   ZDP_0084a <chr>, ZDP_0061a <chr>, ZDP_0561a <chr>, ZDP_0057a <chr>,
## #   ZDP_0570a <chr>, ZDP_0899b <chr>, ZDP_0888b <chr>, ZDP_1104a <chr>,
## #   ZDP_1162a <chr>, S1579 <chr>, ZDP_0634a <chr>, S1843 <chr>,
## #   ZDP_0689b <chr>, ZDP_0733b <chr>, ZDP_0758b <chr>, ZDP_1003b <chr>,
## #   ZDP_1012a <chr>, ZDP_0640b <chr>, ZDP_0153a <chr>, ZDP_0154a <chr>,
## #   ZDP_1020a <chr>, S1822 <chr>, ZDP_0994a <chr>, ZDP_1208a <chr>,
## #   ZDP_1489a <chr>, ZDP_0123b <chr>, ZDP_1010a <chr>, ZDP_0001a <chr>,
## #   ZDP_0174a <chr>, ZDP_0178a <chr>, ZDP_1221b <chr>, ZDP_1481a <chr>,
## #   ZDP_0988a <chr>, ZDP_0456a <chr>, ZDP_0368b <chr>, ZDP_0376a <chr>,
## #   BKN_003 <chr>, BKN_004 <chr>, ZDP_0396a <chr>, ZDP_0417a <chr>,
## #   ZDP_0563a <chr>, ZDP_0727a <chr>, ZDP_0746b <chr>, ZDP_0756b <chr>,
## #   ZDP_1126a <chr>, ZDP_1186a <chr>, ZDP_1100a <chr>, ...
\end{verbatim}

\begin{Shaded}
\begin{Highlighting}[]
\NormalTok{SNP_tTeosinte<-}\StringTok{ }\KeywordTok{inner_join}\NormalTok{(RedSNP,tTeosinte, }\DataTypeTok{by=} \StringTok{"SNP_ID"}\NormalTok{) }\CommentTok{#method #2}
\NormalTok{SNP_tTeosinte}
\end{Highlighting}
\end{Shaded}

\begin{verbatim}
## # A tibble: 983 x 979
##    SNP_ID   Chromosome Position  S0881 S1057 S1087 S1689 S1697 S1698 S1703
##    <chr>    <chr>      <chr>     <chr> <chr> <chr> <chr> <chr> <chr> <chr>
##  1 abph1.20 2          27403404  C/G   C/G   G/G   G/G   C/G   C/G   G/G  
##  2 abph1.22 2          27403892  A/A   A/A   A/A   A/A   A/A   A/A   A/A  
##  3 ae1.3    5          167889790 T/T   G/T   T/T   T/T   T/T   T/T   T/T  
##  4 ae1.4    5          167889682 G/G   A/G   G/G   G/G   G/G   A/G   G/G  
##  5 ae1.5    5          167889821 T/T   T/T   T/T   T/T   T/T   T/T   T/T  
##  6 an1.4    1          240498509 C/C   C/C   C/C   C/C   C/C   C/C   C/C  
##  7 ba1.6    3          181362952 A/G   A/A   A/A   A/A   A/A   A/A   A/A  
##  8 ba1.9    3          181362666 G/G   G/G   G/G   G/G   G/G   G/G   G/G  
##  9 bt2.5    4          66290049  T/T   C/T   C/T   C/C   C/C   C/T   C/C  
## 10 bt2.7    4          66290994  A/A   A/A   A/A   A/A   A/A   A/A   A/A  
## # ... with 973 more rows, and 969 more variables: S1704 <chr>,
## #   TA057 <chr>, TA137 <chr>, TA153 <chr>, TA205 <chr>, TA217 <chr>,
## #   TA242 <chr>, TA285 <chr>, TA293 <chr>, TAMex0091.1 <chr>,
## #   TAMex0153.1 <chr>, TAMex0238.1 <chr>, TAMex0396.1 <chr>,
## #   TAMex0446.1 <chr>, TAMex0455.2 <chr>, TAMex0520.1 <chr>,
## #   TAMex0608.5 <chr>, TAMex0624.1 <chr>, TAMex0643.1 <chr>,
## #   TAMex0775.2 <chr>, TAMex0930.1 <chr>, TAMex1083.1 <chr>,
## #   TAMex1111.1 <chr>, TAMex1195.2 <chr>, TAMex1222.2 <chr>,
## #   TAMex1308.1 <chr>, TAMex1321.1 <chr>, TAMex1344.2 <chr>,
## #   TAMex1422.2 <chr>, TAMex1488.1 <chr>, TAMex1561.2 <chr>,
## #   TAMex0005.1 <chr>, TAMex0077.1 <chr>, TAMex0240.2 <chr>,
## #   TAMex0299.1 <chr>, TAMex0382.2 <chr>, TAMex0482.2 <chr>,
## #   TAMex0483.1 <chr>, TAMex0488.1 <chr>, TAMex0509.1 <chr>,
## #   TAMex0621.2 <chr>, TAMex0819.1 <chr>, TAMex0840.1 <chr>,
## #   TAMex0855.2 <chr>, TAMex0901.2 <chr>, TAMex0925.1 <chr>,
## #   TAMex0942.1 <chr>, TAMex0966.1 <chr>, TAMex1005.1 <chr>,
## #   TAMex1136.1 <chr>, TAMex1165.1 <chr>, TAMex1340.1 <chr>,
## #   TAMex1349.2 <chr>, TAMex1391.2 <chr>, TAMex1437.1 <chr>,
## #   TAMex1508.4 <chr>, TAMex0050.1 <chr>, TAMex0106.1 <chr>,
## #   TAMex0167.1 <chr>, TAMex0181.2 <chr>, TAMex0207.2 <chr>,
## #   TAMex0438.1 <chr>, TAMex0515.1 <chr>, TAMex0517.1 <chr>,
## #   TAMex0568.1 <chr>, TAMex0645.1 <chr>, TAMex0667.1 <chr>,
## #   TAMex0757.1 <chr>, TAMex0845.2 <chr>, TAMex0849.3 <chr>,
## #   TAMex0918.2 <chr>, TAMex0947.1 <chr>, TAMex0982.1 <chr>,
## #   TAMex1139.2 <chr>, TAMex1298.4 <chr>, TAMex1348.3 <chr>,
## #   TAMex1358.1 <chr>, TAMex1439.2 <chr>, TAMex1443.1 <chr>,
## #   TAMex1467.1 <chr>, TAMex0267.2 <chr>, TAMex0280.3 <chr>,
## #   TAMex0400.1 <chr>, TAMex0403.1 <chr>, TAMex0412.1 <chr>,
## #   TAMex0495.1 <chr>, TAMex0514.1 <chr>, TAMex0544.1 <chr>,
## #   TAMex0587.1 <chr>, TAMex0613.2 <chr>, TAMex0766.2 <chr>,
## #   TAMex0785.1 <chr>, TAMex0874.1 <chr>, TAMex0897.2 <chr>,
## #   TAMex1069.2 <chr>, TAMex1087.1 <chr>, TAMex1107.1 <chr>,
## #   TAMex1160.1 <chr>, TAMex1178.1 <chr>, TAMex1246.3 <chr>, ...
\end{verbatim}

\subsection{Trim files to remove multiple or unknown positions from the
files}\label{trim-files-to-remove-multiple-or-unknown-positions-from-the-files}

\begin{Shaded}
\begin{Highlighting}[]
\NormalTok{SNP_tMaize<-}\StringTok{ }\KeywordTok{filter}\NormalTok{(SNP_tMaize, Chromosome }\OperatorTok{!=}\StringTok{ 'multiple'}\NormalTok{, Chromosome }\OperatorTok{!=}\StringTok{ 'unknown'}\NormalTok{, Position }\OperatorTok{!=}\StringTok{ 'multiple'}\NormalTok{)}
\NormalTok{SNP_tMaize}\OperatorTok{$}\NormalTok{Chromosome <-}\StringTok{ }\KeywordTok{as.integer}\NormalTok{(SNP_tMaize}\OperatorTok{$}\NormalTok{Chromosome)}
\NormalTok{SNP_tMaize}\OperatorTok{$}\NormalTok{Position <-}\StringTok{ }\KeywordTok{as.integer}\NormalTok{(SNP_tMaize}\OperatorTok{$}\NormalTok{Position)}
\NormalTok{SNP_tMaize}
\end{Highlighting}
\end{Shaded}

\begin{verbatim}
## # A tibble: 939 x 1,577
##    SNP_ID   Chromosome  Position ZDP_0752a ZDP_0793a ZDP_0612a ZDP_0602a
##    <chr>         <int>     <int> <chr>     <chr>     <chr>     <chr>    
##  1 abph1.20          2  27403404 C/G       C/G       C/C       C/G      
##  2 abph1.22          2  27403892 A/A       A/A       A/A       A/A      
##  3 ae1.3             5 167889790 T/T       T/T       T/T       G/T      
##  4 ae1.4             5 167889682 G/G       G/G       G/G       A/G      
##  5 ae1.5             5 167889821 C/C       C/T       C/C       C/T      
##  6 an1.4             1 240498509 C/G       C/G       C/C       C/C      
##  7 ba1.6             3 181362952 G/G       A/G       G/G       G/G      
##  8 ba1.9             3 181362666 G/G       G/T       ?/?       G/G      
##  9 bt2.5             4  66290049 C/C       C/C       C/C       C/C      
## 10 bt2.7             4  66290994 ?/?       ?/?       G/G       G/G      
## # ... with 929 more rows, and 1,570 more variables: ZDP_0581a <chr>,
## #   ZDP_0552a <chr>, ZDP_0543a <chr>, ZDP_0042a <chr>, ZDP_0121a <chr>,
## #   ZDP_0101b <chr>, ZDP_0087a <chr>, ZDP_0033a <chr>, ZDP_0373b <chr>,
## #   ZDP_0168a <chr>, ZDP_0169a <chr>, ZDP_0517b <chr>, ZDP_1098a <chr>,
## #   ZDP_0591a <chr>, ZDP_0258a <chr>, ZDP_0009a <chr>, ZDP_0016a <chr>,
## #   ZDP_0010a <chr>, ZDP_0018a <chr>, ZDP_0647b <chr>, ZDP_1064b <chr>,
## #   ZDP_1053a <chr>, ZDP_1445b <chr>, ZDP_0163a <chr>, ZDP_0164a <chr>,
## #   ZDP_0166a <chr>, ZDP_0839a <chr>, ZDP_0849a <chr>, ZDP_0858b <chr>,
## #   ZDP_0184b <chr>, ZDP_0179a <chr>, ZDP_0180a <chr>, ZDP_1021a <chr>,
## #   ZDP_0181a <chr>, ZDP_1004a <chr>, S1799 <chr>, ZDP_0856a <chr>,
## #   ZDP_0915b <chr>, ZDP_1194b <chr>, ZDP_1476a <chr>, ZDP_0935a <chr>,
## #   ZDP_0197a <chr>, ZDP_0206a <chr>, ZDP_0195a <chr>, ZDP_0051a <chr>,
## #   ZDP_0054a <chr>, ZDP_0052a <chr>, ZDP_0675b <chr>, ZDP_0791a <chr>,
## #   ZDP_0814a <chr>, ZDP_0122a <chr>, ZDP_0090a <chr>, ZDP_0083a <chr>,
## #   ZDP_0084a <chr>, ZDP_0061a <chr>, ZDP_0561a <chr>, ZDP_0057a <chr>,
## #   ZDP_0570a <chr>, ZDP_0899b <chr>, ZDP_0888b <chr>, ZDP_1104a <chr>,
## #   ZDP_1162a <chr>, S1579 <chr>, ZDP_0634a <chr>, S1843 <chr>,
## #   ZDP_0689b <chr>, ZDP_0733b <chr>, ZDP_0758b <chr>, ZDP_1003b <chr>,
## #   ZDP_1012a <chr>, ZDP_0640b <chr>, ZDP_0153a <chr>, ZDP_0154a <chr>,
## #   ZDP_1020a <chr>, S1822 <chr>, ZDP_0994a <chr>, ZDP_1208a <chr>,
## #   ZDP_1489a <chr>, ZDP_0123b <chr>, ZDP_1010a <chr>, ZDP_0001a <chr>,
## #   ZDP_0174a <chr>, ZDP_0178a <chr>, ZDP_1221b <chr>, ZDP_1481a <chr>,
## #   ZDP_0988a <chr>, ZDP_0456a <chr>, ZDP_0368b <chr>, ZDP_0376a <chr>,
## #   BKN_003 <chr>, BKN_004 <chr>, ZDP_0396a <chr>, ZDP_0417a <chr>,
## #   ZDP_0563a <chr>, ZDP_0727a <chr>, ZDP_0746b <chr>, ZDP_0756b <chr>,
## #   ZDP_1126a <chr>, ZDP_1186a <chr>, ZDP_1100a <chr>, ...
\end{verbatim}

\begin{Shaded}
\begin{Highlighting}[]
\NormalTok{SNP_tTeosinte<-}\StringTok{ }\KeywordTok{filter}\NormalTok{(SNP_tTeosinte, Chromosome }\OperatorTok{!=}\StringTok{ 'multiple'}\NormalTok{, Chromosome }\OperatorTok{!=}\StringTok{ 'unknown'}\NormalTok{, Position }\OperatorTok{!=}\StringTok{ 'multiple'}\NormalTok{)}
\NormalTok{SNP_tTeosinte}\OperatorTok{$}\NormalTok{Chromosome <-}\StringTok{ }\KeywordTok{as.integer}\NormalTok{(SNP_tTeosinte}\OperatorTok{$}\NormalTok{Chromosome)}
\NormalTok{SNP_tTeosinte}\OperatorTok{$}\NormalTok{Position <-}\StringTok{ }\KeywordTok{as.integer}\NormalTok{(SNP_tTeosinte}\OperatorTok{$}\NormalTok{Position)}
\NormalTok{SNP_tTeosinte}
\end{Highlighting}
\end{Shaded}

\begin{verbatim}
## # A tibble: 939 x 979
##    SNP_ID   Chromosome  Position S0881 S1057 S1087 S1689 S1697 S1698 S1703
##    <chr>         <int>     <int> <chr> <chr> <chr> <chr> <chr> <chr> <chr>
##  1 abph1.20          2  27403404 C/G   C/G   G/G   G/G   C/G   C/G   G/G  
##  2 abph1.22          2  27403892 A/A   A/A   A/A   A/A   A/A   A/A   A/A  
##  3 ae1.3             5 167889790 T/T   G/T   T/T   T/T   T/T   T/T   T/T  
##  4 ae1.4             5 167889682 G/G   A/G   G/G   G/G   G/G   A/G   G/G  
##  5 ae1.5             5 167889821 T/T   T/T   T/T   T/T   T/T   T/T   T/T  
##  6 an1.4             1 240498509 C/C   C/C   C/C   C/C   C/C   C/C   C/C  
##  7 ba1.6             3 181362952 A/G   A/A   A/A   A/A   A/A   A/A   A/A  
##  8 ba1.9             3 181362666 G/G   G/G   G/G   G/G   G/G   G/G   G/G  
##  9 bt2.5             4  66290049 T/T   C/T   C/T   C/C   C/C   C/T   C/C  
## 10 bt2.7             4  66290994 A/A   A/A   A/A   A/A   A/A   A/A   A/A  
## # ... with 929 more rows, and 969 more variables: S1704 <chr>,
## #   TA057 <chr>, TA137 <chr>, TA153 <chr>, TA205 <chr>, TA217 <chr>,
## #   TA242 <chr>, TA285 <chr>, TA293 <chr>, TAMex0091.1 <chr>,
## #   TAMex0153.1 <chr>, TAMex0238.1 <chr>, TAMex0396.1 <chr>,
## #   TAMex0446.1 <chr>, TAMex0455.2 <chr>, TAMex0520.1 <chr>,
## #   TAMex0608.5 <chr>, TAMex0624.1 <chr>, TAMex0643.1 <chr>,
## #   TAMex0775.2 <chr>, TAMex0930.1 <chr>, TAMex1083.1 <chr>,
## #   TAMex1111.1 <chr>, TAMex1195.2 <chr>, TAMex1222.2 <chr>,
## #   TAMex1308.1 <chr>, TAMex1321.1 <chr>, TAMex1344.2 <chr>,
## #   TAMex1422.2 <chr>, TAMex1488.1 <chr>, TAMex1561.2 <chr>,
## #   TAMex0005.1 <chr>, TAMex0077.1 <chr>, TAMex0240.2 <chr>,
## #   TAMex0299.1 <chr>, TAMex0382.2 <chr>, TAMex0482.2 <chr>,
## #   TAMex0483.1 <chr>, TAMex0488.1 <chr>, TAMex0509.1 <chr>,
## #   TAMex0621.2 <chr>, TAMex0819.1 <chr>, TAMex0840.1 <chr>,
## #   TAMex0855.2 <chr>, TAMex0901.2 <chr>, TAMex0925.1 <chr>,
## #   TAMex0942.1 <chr>, TAMex0966.1 <chr>, TAMex1005.1 <chr>,
## #   TAMex1136.1 <chr>, TAMex1165.1 <chr>, TAMex1340.1 <chr>,
## #   TAMex1349.2 <chr>, TAMex1391.2 <chr>, TAMex1437.1 <chr>,
## #   TAMex1508.4 <chr>, TAMex0050.1 <chr>, TAMex0106.1 <chr>,
## #   TAMex0167.1 <chr>, TAMex0181.2 <chr>, TAMex0207.2 <chr>,
## #   TAMex0438.1 <chr>, TAMex0515.1 <chr>, TAMex0517.1 <chr>,
## #   TAMex0568.1 <chr>, TAMex0645.1 <chr>, TAMex0667.1 <chr>,
## #   TAMex0757.1 <chr>, TAMex0845.2 <chr>, TAMex0849.3 <chr>,
## #   TAMex0918.2 <chr>, TAMex0947.1 <chr>, TAMex0982.1 <chr>,
## #   TAMex1139.2 <chr>, TAMex1298.4 <chr>, TAMex1348.3 <chr>,
## #   TAMex1358.1 <chr>, TAMex1439.2 <chr>, TAMex1443.1 <chr>,
## #   TAMex1467.1 <chr>, TAMex0267.2 <chr>, TAMex0280.3 <chr>,
## #   TAMex0400.1 <chr>, TAMex0403.1 <chr>, TAMex0412.1 <chr>,
## #   TAMex0495.1 <chr>, TAMex0514.1 <chr>, TAMex0544.1 <chr>,
## #   TAMex0587.1 <chr>, TAMex0613.2 <chr>, TAMex0766.2 <chr>,
## #   TAMex0785.1 <chr>, TAMex0874.1 <chr>, TAMex0897.2 <chr>,
## #   TAMex1069.2 <chr>, TAMex1087.1 <chr>, TAMex1107.1 <chr>,
## #   TAMex1160.1 <chr>, TAMex1178.1 <chr>, TAMex1246.3 <chr>, ...
\end{verbatim}

\subsection{\texorpdfstring{Sort the merged \& trimmed files by
chromosomes and positions by
`arrange()'}{Sort the merged \& trimmed files by chromosomes and positions by arrange()}}\label{sort-the-merged-trimmed-files-by-chromosomes-and-positions-by-arrange}

\begin{Shaded}
\begin{Highlighting}[]
\NormalTok{SNP_tMaize_sorted<-}\KeywordTok{arrange}\NormalTok{(SNP_tMaize, Chromosome, Position)}
\KeywordTok{tail}\NormalTok{(SNP_tMaize_sorted)}
\end{Highlighting}
\end{Shaded}

\begin{verbatim}
## # A tibble: 6 x 1,577
##   SNP_ID      Chromosome  Position ZDP_0752a ZDP_0793a ZDP_0612a ZDP_0602a
##   <chr>            <int>     <int> <chr>     <chr>     <chr>     <chr>    
## 1 PZB00055.1          10 140493174 C/C       C/C       C/G       C/C      
## 2 PZA00727.11         10 142854665 C/C       C/T       C/C       C/C      
## 3 PZA00727.12         10 142854744 A/A       A/G       G/G       A/A      
## 4 PZA02969.11         10 143676107 C/C       ?/?       C/C       C/C      
## 5 PZD00075.1          10 146599154 C/G       C/C       C/G       C/G      
## 6 PZD00075.2          10 146599484 C/C       C/C       C/C       C/C      
## # ... with 1,570 more variables: ZDP_0581a <chr>, ZDP_0552a <chr>,
## #   ZDP_0543a <chr>, ZDP_0042a <chr>, ZDP_0121a <chr>, ZDP_0101b <chr>,
## #   ZDP_0087a <chr>, ZDP_0033a <chr>, ZDP_0373b <chr>, ZDP_0168a <chr>,
## #   ZDP_0169a <chr>, ZDP_0517b <chr>, ZDP_1098a <chr>, ZDP_0591a <chr>,
## #   ZDP_0258a <chr>, ZDP_0009a <chr>, ZDP_0016a <chr>, ZDP_0010a <chr>,
## #   ZDP_0018a <chr>, ZDP_0647b <chr>, ZDP_1064b <chr>, ZDP_1053a <chr>,
## #   ZDP_1445b <chr>, ZDP_0163a <chr>, ZDP_0164a <chr>, ZDP_0166a <chr>,
## #   ZDP_0839a <chr>, ZDP_0849a <chr>, ZDP_0858b <chr>, ZDP_0184b <chr>,
## #   ZDP_0179a <chr>, ZDP_0180a <chr>, ZDP_1021a <chr>, ZDP_0181a <chr>,
## #   ZDP_1004a <chr>, S1799 <chr>, ZDP_0856a <chr>, ZDP_0915b <chr>,
## #   ZDP_1194b <chr>, ZDP_1476a <chr>, ZDP_0935a <chr>, ZDP_0197a <chr>,
## #   ZDP_0206a <chr>, ZDP_0195a <chr>, ZDP_0051a <chr>, ZDP_0054a <chr>,
## #   ZDP_0052a <chr>, ZDP_0675b <chr>, ZDP_0791a <chr>, ZDP_0814a <chr>,
## #   ZDP_0122a <chr>, ZDP_0090a <chr>, ZDP_0083a <chr>, ZDP_0084a <chr>,
## #   ZDP_0061a <chr>, ZDP_0561a <chr>, ZDP_0057a <chr>, ZDP_0570a <chr>,
## #   ZDP_0899b <chr>, ZDP_0888b <chr>, ZDP_1104a <chr>, ZDP_1162a <chr>,
## #   S1579 <chr>, ZDP_0634a <chr>, S1843 <chr>, ZDP_0689b <chr>,
## #   ZDP_0733b <chr>, ZDP_0758b <chr>, ZDP_1003b <chr>, ZDP_1012a <chr>,
## #   ZDP_0640b <chr>, ZDP_0153a <chr>, ZDP_0154a <chr>, ZDP_1020a <chr>,
## #   S1822 <chr>, ZDP_0994a <chr>, ZDP_1208a <chr>, ZDP_1489a <chr>,
## #   ZDP_0123b <chr>, ZDP_1010a <chr>, ZDP_0001a <chr>, ZDP_0174a <chr>,
## #   ZDP_0178a <chr>, ZDP_1221b <chr>, ZDP_1481a <chr>, ZDP_0988a <chr>,
## #   ZDP_0456a <chr>, ZDP_0368b <chr>, ZDP_0376a <chr>, BKN_003 <chr>,
## #   BKN_004 <chr>, ZDP_0396a <chr>, ZDP_0417a <chr>, ZDP_0563a <chr>,
## #   ZDP_0727a <chr>, ZDP_0746b <chr>, ZDP_0756b <chr>, ZDP_1126a <chr>,
## #   ZDP_1186a <chr>, ZDP_1100a <chr>, ...
\end{verbatim}

\begin{Shaded}
\begin{Highlighting}[]
\NormalTok{SNP_tTeosinte_sorted<-}\KeywordTok{arrange}\NormalTok{(SNP_tTeosinte, Chromosome, Position)}
\KeywordTok{tail}\NormalTok{(SNP_tTeosinte_sorted)}
\end{Highlighting}
\end{Shaded}

\begin{verbatim}
## # A tibble: 6 x 979
##   SNP_ID     Chromosome Position S0881 S1057 S1087 S1689 S1697 S1698 S1703
##   <chr>           <int>    <int> <chr> <chr> <chr> <chr> <chr> <chr> <chr>
## 1 PZB00055.1         10   1.40e8 G/G   C/G   G/G   C/G   C/G   C/G   C/C  
## 2 PZA00727.~         10   1.43e8 C/C   C/C   C/T   C/T   C/T   C/C   C/C  
## 3 PZA00727.~         10   1.43e8 A/G   A/A   A/A   A/A   A/G   G/G   A/A  
## 4 PZA02969.~         10   1.44e8 C/C   C/C   C/C   C/C   C/C   C/T   C/C  
## 5 PZD00075.1         10   1.47e8 C/G   C/G   G/G   C/C   C/C   C/G   C/C  
## 6 PZD00075.2         10   1.47e8 C/C   C/C   ?/?   C/C   ?/?   C/C   C/C  
## # ... with 969 more variables: S1704 <chr>, TA057 <chr>, TA137 <chr>,
## #   TA153 <chr>, TA205 <chr>, TA217 <chr>, TA242 <chr>, TA285 <chr>,
## #   TA293 <chr>, TAMex0091.1 <chr>, TAMex0153.1 <chr>, TAMex0238.1 <chr>,
## #   TAMex0396.1 <chr>, TAMex0446.1 <chr>, TAMex0455.2 <chr>,
## #   TAMex0520.1 <chr>, TAMex0608.5 <chr>, TAMex0624.1 <chr>,
## #   TAMex0643.1 <chr>, TAMex0775.2 <chr>, TAMex0930.1 <chr>,
## #   TAMex1083.1 <chr>, TAMex1111.1 <chr>, TAMex1195.2 <chr>,
## #   TAMex1222.2 <chr>, TAMex1308.1 <chr>, TAMex1321.1 <chr>,
## #   TAMex1344.2 <chr>, TAMex1422.2 <chr>, TAMex1488.1 <chr>,
## #   TAMex1561.2 <chr>, TAMex0005.1 <chr>, TAMex0077.1 <chr>,
## #   TAMex0240.2 <chr>, TAMex0299.1 <chr>, TAMex0382.2 <chr>,
## #   TAMex0482.2 <chr>, TAMex0483.1 <chr>, TAMex0488.1 <chr>,
## #   TAMex0509.1 <chr>, TAMex0621.2 <chr>, TAMex0819.1 <chr>,
## #   TAMex0840.1 <chr>, TAMex0855.2 <chr>, TAMex0901.2 <chr>,
## #   TAMex0925.1 <chr>, TAMex0942.1 <chr>, TAMex0966.1 <chr>,
## #   TAMex1005.1 <chr>, TAMex1136.1 <chr>, TAMex1165.1 <chr>,
## #   TAMex1340.1 <chr>, TAMex1349.2 <chr>, TAMex1391.2 <chr>,
## #   TAMex1437.1 <chr>, TAMex1508.4 <chr>, TAMex0050.1 <chr>,
## #   TAMex0106.1 <chr>, TAMex0167.1 <chr>, TAMex0181.2 <chr>,
## #   TAMex0207.2 <chr>, TAMex0438.1 <chr>, TAMex0515.1 <chr>,
## #   TAMex0517.1 <chr>, TAMex0568.1 <chr>, TAMex0645.1 <chr>,
## #   TAMex0667.1 <chr>, TAMex0757.1 <chr>, TAMex0845.2 <chr>,
## #   TAMex0849.3 <chr>, TAMex0918.2 <chr>, TAMex0947.1 <chr>,
## #   TAMex0982.1 <chr>, TAMex1139.2 <chr>, TAMex1298.4 <chr>,
## #   TAMex1348.3 <chr>, TAMex1358.1 <chr>, TAMex1439.2 <chr>,
## #   TAMex1443.1 <chr>, TAMex1467.1 <chr>, TAMex0267.2 <chr>,
## #   TAMex0280.3 <chr>, TAMex0400.1 <chr>, TAMex0403.1 <chr>,
## #   TAMex0412.1 <chr>, TAMex0495.1 <chr>, TAMex0514.1 <chr>,
## #   TAMex0544.1 <chr>, TAMex0587.1 <chr>, TAMex0613.2 <chr>,
## #   TAMex0766.2 <chr>, TAMex0785.1 <chr>, TAMex0874.1 <chr>,
## #   TAMex0897.2 <chr>, TAMex1069.2 <chr>, TAMex1087.1 <chr>,
## #   TAMex1107.1 <chr>, TAMex1160.1 <chr>, TAMex1178.1 <chr>,
## #   TAMex1246.3 <chr>, ...
\end{verbatim}

\subsection{\texorpdfstring{Seperating files by chromosomes by
increasing orders of positions, missing value by
`?'}{Seperating files by chromosomes by increasing orders of positions, missing value by ?}}\label{seperating-files-by-chromosomes-by-increasing-orders-of-positions-missing-value-by}

\begin{Shaded}
\begin{Highlighting}[]
\ControlFlowTok{for}\NormalTok{ (i }\ControlFlowTok{in} \DecValTok{1}\OperatorTok{:}\DecValTok{10}\NormalTok{)\{}
\NormalTok{  filename <-}\StringTok{ }\KeywordTok{paste}\NormalTok{(}\StringTok{"../Output/"}\NormalTok{, }\StringTok{"Maize_inc_chr"}\NormalTok{, i, }\StringTok{".txt"}\NormalTok{, }\DataTypeTok{sep =} \StringTok{""}\NormalTok{)}
  \KeywordTok{write_tsv}\NormalTok{(}\KeywordTok{filter}\NormalTok{(SNP_tMaize_sorted, Chromosome }\OperatorTok{==}\StringTok{ }\NormalTok{i), filename, }\DataTypeTok{col_names =}\NormalTok{ T)}
\NormalTok{\}}

\ControlFlowTok{for}\NormalTok{ (i }\ControlFlowTok{in} \DecValTok{1}\OperatorTok{:}\DecValTok{10}\NormalTok{)\{}
\NormalTok{  filename <-}\StringTok{ }\KeywordTok{paste}\NormalTok{(}\StringTok{"../Output/"}\NormalTok{, }\StringTok{"Teosinte_inc_chr"}\NormalTok{, i, }\StringTok{".txt"}\NormalTok{, }\DataTypeTok{sep =} \StringTok{""}\NormalTok{)}
  \KeywordTok{write_tsv}\NormalTok{(}\KeywordTok{filter}\NormalTok{(SNP_tTeosinte_sorted, Chromosome }\OperatorTok{==}\StringTok{ }\NormalTok{i), filename, }\DataTypeTok{col_names =}\NormalTok{ T)}
\NormalTok{\}}
\end{Highlighting}
\end{Shaded}

\subsection{\texorpdfstring{Seperating files by chromosomes by
increasing orders of positions, missing value by
`-'}{Seperating files by chromosomes by increasing orders of positions, missing value by -}}\label{seperating-files-by-chromosomes-by-increasing-orders-of-positions-missing-value-by--}

\begin{Shaded}
\begin{Highlighting}[]
\CommentTok{# first replace the missing value with "-"}
\NormalTok{SNP_tMaize_sorted[SNP_tMaize_sorted }\OperatorTok{==}\StringTok{ "?/?"}\NormalTok{] <-}\StringTok{ "-/-"}
\NormalTok{SNP_tMaize_sorted}
\end{Highlighting}
\end{Shaded}

\begin{verbatim}
## # A tibble: 939 x 1,577
##    SNP_ID      Chromosome Position ZDP_0752a ZDP_0793a ZDP_0612a ZDP_0602a
##    <chr>            <int>    <int> <chr>     <chr>     <chr>     <chr>    
##  1 PZB00859.1           1   157104 C/C       C/C       A/C       C/C      
##  2 PZA02962.13          1  3205252 T/T       T/T       T/T       T/T      
##  3 PZA00393.1           1  4175293 C/T       C/C       T/T       T/T      
##  4 PZA00393.4           1  4175573 G/G       G/G       G/G       G/G      
##  5 PZA02869.8           1  4429897 C/C       C/C       -/-       -/-      
##  6 PZA02869.2           1  4430055 T/T       T/T       C/T       -/-      
##  7 PZD00021.4           1  4835472 G/G       G/G       G/G       G/G      
##  8 PZD00021.2           1  4835540 T/T       -/-       T/T       T/T      
##  9 PZD00021.5           1  4835596 A/A       A/A       A/A       A/A      
## 10 zagl1.6              1  4835658 T/T       T/T       T/T       T/T      
## # ... with 929 more rows, and 1,570 more variables: ZDP_0581a <chr>,
## #   ZDP_0552a <chr>, ZDP_0543a <chr>, ZDP_0042a <chr>, ZDP_0121a <chr>,
## #   ZDP_0101b <chr>, ZDP_0087a <chr>, ZDP_0033a <chr>, ZDP_0373b <chr>,
## #   ZDP_0168a <chr>, ZDP_0169a <chr>, ZDP_0517b <chr>, ZDP_1098a <chr>,
## #   ZDP_0591a <chr>, ZDP_0258a <chr>, ZDP_0009a <chr>, ZDP_0016a <chr>,
## #   ZDP_0010a <chr>, ZDP_0018a <chr>, ZDP_0647b <chr>, ZDP_1064b <chr>,
## #   ZDP_1053a <chr>, ZDP_1445b <chr>, ZDP_0163a <chr>, ZDP_0164a <chr>,
## #   ZDP_0166a <chr>, ZDP_0839a <chr>, ZDP_0849a <chr>, ZDP_0858b <chr>,
## #   ZDP_0184b <chr>, ZDP_0179a <chr>, ZDP_0180a <chr>, ZDP_1021a <chr>,
## #   ZDP_0181a <chr>, ZDP_1004a <chr>, S1799 <chr>, ZDP_0856a <chr>,
## #   ZDP_0915b <chr>, ZDP_1194b <chr>, ZDP_1476a <chr>, ZDP_0935a <chr>,
## #   ZDP_0197a <chr>, ZDP_0206a <chr>, ZDP_0195a <chr>, ZDP_0051a <chr>,
## #   ZDP_0054a <chr>, ZDP_0052a <chr>, ZDP_0675b <chr>, ZDP_0791a <chr>,
## #   ZDP_0814a <chr>, ZDP_0122a <chr>, ZDP_0090a <chr>, ZDP_0083a <chr>,
## #   ZDP_0084a <chr>, ZDP_0061a <chr>, ZDP_0561a <chr>, ZDP_0057a <chr>,
## #   ZDP_0570a <chr>, ZDP_0899b <chr>, ZDP_0888b <chr>, ZDP_1104a <chr>,
## #   ZDP_1162a <chr>, S1579 <chr>, ZDP_0634a <chr>, S1843 <chr>,
## #   ZDP_0689b <chr>, ZDP_0733b <chr>, ZDP_0758b <chr>, ZDP_1003b <chr>,
## #   ZDP_1012a <chr>, ZDP_0640b <chr>, ZDP_0153a <chr>, ZDP_0154a <chr>,
## #   ZDP_1020a <chr>, S1822 <chr>, ZDP_0994a <chr>, ZDP_1208a <chr>,
## #   ZDP_1489a <chr>, ZDP_0123b <chr>, ZDP_1010a <chr>, ZDP_0001a <chr>,
## #   ZDP_0174a <chr>, ZDP_0178a <chr>, ZDP_1221b <chr>, ZDP_1481a <chr>,
## #   ZDP_0988a <chr>, ZDP_0456a <chr>, ZDP_0368b <chr>, ZDP_0376a <chr>,
## #   BKN_003 <chr>, BKN_004 <chr>, ZDP_0396a <chr>, ZDP_0417a <chr>,
## #   ZDP_0563a <chr>, ZDP_0727a <chr>, ZDP_0746b <chr>, ZDP_0756b <chr>,
## #   ZDP_1126a <chr>, ZDP_1186a <chr>, ZDP_1100a <chr>, ...
\end{verbatim}

\begin{Shaded}
\begin{Highlighting}[]
\CommentTok{# USe pipe to put multiple functions together}
\ControlFlowTok{for}\NormalTok{ (i }\ControlFlowTok{in} \DecValTok{1}\OperatorTok{:}\DecValTok{10}\NormalTok{)\{}
\NormalTok{  filename <-}\StringTok{ }\KeywordTok{paste}\NormalTok{(}\StringTok{"../Output/"}\NormalTok{, }\StringTok{"Maize_dec_chr"}\NormalTok{, i, }\StringTok{".txt"}\NormalTok{, }\DataTypeTok{sep =} \StringTok{""}\NormalTok{) }\CommentTok{#set the path and filename for write_tsv}
\NormalTok{  SNP_tMaize_sorted }\OperatorTok
\StringTok{  }\KeywordTok{filter}\NormalTok{(Chromosome }\OperatorTok{==}\StringTok{ }\NormalTok{i) }\OperatorTok
\StringTok{  }\KeywordTok{arrange}\NormalTok{ (}\KeywordTok{desc}\NormalTok{(Position)) }\OperatorTok
\StringTok{  }\KeywordTok{write_tsv}\NormalTok{(filename, }\DataTypeTok{col_names =}\NormalTok{ T)}
\NormalTok{\}}

\CommentTok{# first replace the missing value with "-"}
\NormalTok{SNP_tTeosinte_sorted[SNP_tTeosinte_sorted }\OperatorTok{==}\StringTok{ "?/?"}\NormalTok{] <-}\StringTok{ "-/-"}
\NormalTok{SNP_tTeosinte_sorted}
\end{Highlighting}
\end{Shaded}

\begin{verbatim}
## # A tibble: 939 x 979
##    SNP_ID    Chromosome Position S0881 S1057 S1087 S1689 S1697 S1698 S1703
##    <chr>          <int>    <int> <chr> <chr> <chr> <chr> <chr> <chr> <chr>
##  1 PZB00859~          1   157104 A/A   A/C   A/A   C/C   A/C   C/C   C/C  
##  2 PZA02962~          1  3205252 T/T   A/A   T/T   A/T   A/T   T/T   T/T  
##  3 PZA00393~          1  4175293 T/T   T/T   T/T   C/T   C/T   C/T   C/T  
##  4 PZA00393~          1  4175573 G/G   A/G   G/G   G/G   G/G   G/G   G/G  
##  5 PZA02869~          1  4429897 C/C   C/C   C/C   C/C   C/C   C/C   C/C  
##  6 PZA02869~          1  4430055 T/T   T/T   T/T   T/T   T/T   T/T   C/C  
##  7 PZD00021~          1  4835472 G/G   G/G   G/G   G/G   G/G   G/G   -/-  
##  8 PZD00021~          1  4835540 C/C   C/C   T/T   C/C   C/T   T/T   C/C  
##  9 PZD00021~          1  4835596 A/T   A/A   A/A   A/A   A/T   A/A   A/A  
## 10 zagl1.6            1  4835658 C/T   C/T   T/T   T/T   C/T   C/T   C/C  
## # ... with 929 more rows, and 969 more variables: S1704 <chr>,
## #   TA057 <chr>, TA137 <chr>, TA153 <chr>, TA205 <chr>, TA217 <chr>,
## #   TA242 <chr>, TA285 <chr>, TA293 <chr>, TAMex0091.1 <chr>,
## #   TAMex0153.1 <chr>, TAMex0238.1 <chr>, TAMex0396.1 <chr>,
## #   TAMex0446.1 <chr>, TAMex0455.2 <chr>, TAMex0520.1 <chr>,
## #   TAMex0608.5 <chr>, TAMex0624.1 <chr>, TAMex0643.1 <chr>,
## #   TAMex0775.2 <chr>, TAMex0930.1 <chr>, TAMex1083.1 <chr>,
## #   TAMex1111.1 <chr>, TAMex1195.2 <chr>, TAMex1222.2 <chr>,
## #   TAMex1308.1 <chr>, TAMex1321.1 <chr>, TAMex1344.2 <chr>,
## #   TAMex1422.2 <chr>, TAMex1488.1 <chr>, TAMex1561.2 <chr>,
## #   TAMex0005.1 <chr>, TAMex0077.1 <chr>, TAMex0240.2 <chr>,
## #   TAMex0299.1 <chr>, TAMex0382.2 <chr>, TAMex0482.2 <chr>,
## #   TAMex0483.1 <chr>, TAMex0488.1 <chr>, TAMex0509.1 <chr>,
## #   TAMex0621.2 <chr>, TAMex0819.1 <chr>, TAMex0840.1 <chr>,
## #   TAMex0855.2 <chr>, TAMex0901.2 <chr>, TAMex0925.1 <chr>,
## #   TAMex0942.1 <chr>, TAMex0966.1 <chr>, TAMex1005.1 <chr>,
## #   TAMex1136.1 <chr>, TAMex1165.1 <chr>, TAMex1340.1 <chr>,
## #   TAMex1349.2 <chr>, TAMex1391.2 <chr>, TAMex1437.1 <chr>,
## #   TAMex1508.4 <chr>, TAMex0050.1 <chr>, TAMex0106.1 <chr>,
## #   TAMex0167.1 <chr>, TAMex0181.2 <chr>, TAMex0207.2 <chr>,
## #   TAMex0438.1 <chr>, TAMex0515.1 <chr>, TAMex0517.1 <chr>,
## #   TAMex0568.1 <chr>, TAMex0645.1 <chr>, TAMex0667.1 <chr>,
## #   TAMex0757.1 <chr>, TAMex0845.2 <chr>, TAMex0849.3 <chr>,
## #   TAMex0918.2 <chr>, TAMex0947.1 <chr>, TAMex0982.1 <chr>,
## #   TAMex1139.2 <chr>, TAMex1298.4 <chr>, TAMex1348.3 <chr>,
## #   TAMex1358.1 <chr>, TAMex1439.2 <chr>, TAMex1443.1 <chr>,
## #   TAMex1467.1 <chr>, TAMex0267.2 <chr>, TAMex0280.3 <chr>,
## #   TAMex0400.1 <chr>, TAMex0403.1 <chr>, TAMex0412.1 <chr>,
## #   TAMex0495.1 <chr>, TAMex0514.1 <chr>, TAMex0544.1 <chr>,
## #   TAMex0587.1 <chr>, TAMex0613.2 <chr>, TAMex0766.2 <chr>,
## #   TAMex0785.1 <chr>, TAMex0874.1 <chr>, TAMex0897.2 <chr>,
## #   TAMex1069.2 <chr>, TAMex1087.1 <chr>, TAMex1107.1 <chr>,
## #   TAMex1160.1 <chr>, TAMex1178.1 <chr>, TAMex1246.3 <chr>, ...
\end{verbatim}

\begin{Shaded}
\begin{Highlighting}[]
\CommentTok{# USe pipe to put multiple functions together}
\ControlFlowTok{for}\NormalTok{ (i }\ControlFlowTok{in} \DecValTok{1}\OperatorTok{:}\DecValTok{10}\NormalTok{)\{}
\NormalTok{  filename <-}\StringTok{ }\KeywordTok{paste}\NormalTok{(}\StringTok{"../Output/"}\NormalTok{, }\StringTok{"Teosinte_dec_chr"}\NormalTok{, i, }\StringTok{".txt"}\NormalTok{, }\DataTypeTok{sep =} \StringTok{""}\NormalTok{) }\CommentTok{#set the path and filename for write_tsv}
\NormalTok{  SNP_tTeosinte_sorted }\OperatorTok
\StringTok{  }\KeywordTok{filter}\NormalTok{(Chromosome }\OperatorTok{==}\StringTok{ }\NormalTok{i) }\OperatorTok
\StringTok{  }\KeywordTok{arrange}\NormalTok{ (}\KeywordTok{desc}\NormalTok{(Position)) }\OperatorTok
\StringTok{  }\KeywordTok{write_tsv}\NormalTok{(filename, }\DataTypeTok{col_names =}\NormalTok{ T)}
\NormalTok{\}}
\end{Highlighting}
\end{Shaded}

\subsubsection{Part II}\label{part-ii}

\subsection{Reshape}\label{reshape}

\begin{Shaded}
\begin{Highlighting}[]
\KeywordTok{library}\NormalTok{(reshape2)}
\end{Highlighting}
\end{Shaded}

\begin{verbatim}
## 
## Attaching package: 'reshape2'
\end{verbatim}

\begin{verbatim}
## The following object is masked from 'package:tidyr':
## 
##     smiths
\end{verbatim}

\begin{Shaded}
\begin{Highlighting}[]
\NormalTok{Geno2 <-}\StringTok{ }\NormalTok{Geno [,}\OperatorTok{-}\DecValTok{2}\NormalTok{]}
\NormalTok{Geno2}
\end{Highlighting}
\end{Shaded}

\begin{verbatim}
## # A tibble: 2,782 x 985
##    Sample_ID Group abph1.20 abph1.22 ae1.3 ae1.4 ae1.5 an1.4 ba1.6 ba1.9
##    <chr>     <chr> <chr>    <chr>    <chr> <chr> <chr> <chr> <chr> <chr>
##  1 SL-15     TRIPS ?/?      ?/?      T/T   G/G   T/T   C/C   ?/?   G/G  
##  2 SL-16     TRIPS ?/?      ?/?      T/T   ?/?   T/T   C/C   A/G   G/G  
##  3 SL-11     TRIPS ?/?      ?/?      T/T   G/G   T/T   ?/?   G/G   G/G  
##  4 SL-12     TRIPS ?/?      ?/?      T/T   G/G   T/T   C/C   G/G   G/G  
##  5 SL-18     TRIPS ?/?      ?/?      T/T   G/G   T/T   C/C   ?/?   G/G  
##  6 SL-2      TRIPS ?/?      ?/?      T/T   G/G   T/T   C/C   A/G   G/G  
##  7 SL-4      TRIPS ?/?      ?/?      T/T   G/G   T/T   C/C   G/G   G/G  
##  8 SL-6      TRIPS ?/?      ?/?      T/T   G/G   T/T   C/C   ?/?   G/G  
##  9 SL-7      TRIPS ?/?      ?/?      T/T   G/G   T/T   C/C   G/G   G/G  
## 10 SL-5      TRIPS ?/?      ?/?      T/T   G/G   T/T   C/C   ?/?   G/G  
## # ... with 2,772 more rows, and 975 more variables: bt2.5 <chr>,
## #   bt2.7 <chr>, bt2.8 <chr>, Fea2.1 <chr>, Fea2.5 <chr>, id1.3 <chr>,
## #   lg2.11 <chr>, lg2.2 <chr>, pbf1.1 <chr>, pbf1.2 <chr>, pbf1.3 <chr>,
## #   pbf1.5 <chr>, pbf1.6 <chr>, pbf1.7 <chr>, pbf1.8 <chr>,
## #   PZA00003.11 <chr>, PZA00004.2 <chr>, PZA00005.8 <chr>,
## #   PZA00005.9 <chr>, PZA00006.13 <chr>, PZA00006.14 <chr>,
## #   PZA00008.1 <chr>, PZA00010.5 <chr>, PZA00013.10 <chr>,
## #   PZA00013.11 <chr>, PZA00013.9 <chr>, PZA00015.4 <chr>,
## #   PZA00017.1 <chr>, PZA00018.5 <chr>, PZA00029.11 <chr>,
## #   PZA00029.12 <chr>, PZA00030.11 <chr>, PZA00031.5 <chr>,
## #   PZA00041.3 <chr>, PZA00042.2 <chr>, PZA00042.5 <chr>,
## #   PZA00043.7 <chr>, PZA00045.1 <chr>, PZA00047.2 <chr>,
## #   PZA00049.12 <chr>, PZA00050.9 <chr>, PZA00051.2 <chr>,
## #   PZA00058.5 <chr>, PZA00058.6 <chr>, PZA00060.2 <chr>,
## #   PZA00061.1 <chr>, PZA00065.2 <chr>, PZA00069.4 <chr>,
## #   PZA00070.5 <chr>, PZA00078.2 <chr>, PZA00079.1 <chr>,
## #   PZA00081.17 <chr>, PZA00084.2 <chr>, PZA00084.3 <chr>,
## #   PZA00086.8 <chr>, PZA00088.3 <chr>, PZA00090.2 <chr>,
## #   PZA00092.1 <chr>, PZA00092.5 <chr>, PZA00093.2 <chr>,
## #   PZA00096.26 <chr>, PZA00097.13 <chr>, PZA00098.14 <chr>,
## #   PZA00100.10 <chr>, PZA00100.12 <chr>, PZA00100.14 <chr>,
## #   PZA00100.9 <chr>, PZA00103.20 <chr>, PZA00106.9 <chr>,
## #   PZA00107.18 <chr>, PZA00108.12 <chr>, PZA00108.14 <chr>,
## #   PZA00108.15 <chr>, PZA00109.3 <chr>, PZA00109.5 <chr>,
## #   PZA00111.2 <chr>, PZA00111.4 <chr>, PZA00111.5 <chr>,
## #   PZA00111.6 <chr>, PZA00111.8 <chr>, PZA00114.3 <chr>,
## #   PZA00116.2 <chr>, PZA00119.4 <chr>, PZA00120.4 <chr>,
## #   PZA00123.1 <chr>, PZA00125.2 <chr>, PZA00131.14 <chr>,
## #   PZA00132.17 <chr>, PZA00132.18 <chr>, PZA00132.3 <chr>,
## #   PZA00135.6 <chr>, PZA00137.2 <chr>, PZA00139.14 <chr>,
## #   PZA00140.10 <chr>, PZA00140.6 <chr>, PZA00140.9 <chr>,
## #   PZA00142.6 <chr>, PZA00148.2 <chr>, PZA00153.3 <chr>,
## #   PZA00153.6 <chr>, ...
\end{verbatim}

\begin{Shaded}
\begin{Highlighting}[]
\NormalTok{Geno2_melt <-}\StringTok{ }\KeywordTok{melt}\NormalTok{(Geno2, }\DataTypeTok{id =} \KeywordTok{c}\NormalTok{(}\StringTok{"Sample_ID"}\NormalTok{, }\StringTok{"Group"}\NormalTok{))}
\KeywordTok{colnames}\NormalTok{(Geno2_melt)[}\DecValTok{3}\NormalTok{] <-}\StringTok{ "SNP_ID"}
\KeywordTok{colnames}\NormalTok{(Geno2_melt)[}\DecValTok{4}\NormalTok{] <-}\StringTok{ "SNP_call"}
\KeywordTok{head}\NormalTok{(Geno2_melt)}
\end{Highlighting}
\end{Shaded}

\begin{verbatim}
##   Sample_ID Group   SNP_ID SNP_call
## 1     SL-15 TRIPS abph1.20      ?/?
## 2     SL-16 TRIPS abph1.20      ?/?
## 3     SL-11 TRIPS abph1.20      ?/?
## 4     SL-12 TRIPS abph1.20      ?/?
## 5     SL-18 TRIPS abph1.20      ?/?
## 6      SL-2 TRIPS abph1.20      ?/?
\end{verbatim}

\begin{Shaded}
\begin{Highlighting}[]
\NormalTok{Geno2_melt_SNPinfo <-}\StringTok{ }\KeywordTok{merge}\NormalTok{(Geno2_melt,RedSNP, }\DataTypeTok{by =} \StringTok{"SNP_ID"}\NormalTok{)}
\NormalTok{Geno2_melt_SNPinfo <-}\StringTok{ }\KeywordTok{filter}\NormalTok{(Geno2_melt_SNPinfo, Chromosome }\OperatorTok{!=}\StringTok{ 'multiple'}\NormalTok{, Chromosome }\OperatorTok{!=}\StringTok{ 'unknown'}\NormalTok{, Position }\OperatorTok{!=}\StringTok{ 'multiple'}\NormalTok{)}

\NormalTok{Geno2_melt_SNPinfo}\OperatorTok{$}\NormalTok{Chromosome =}\StringTok{ }\KeywordTok{as.integer}\NormalTok{(Geno2_melt_SNPinfo}\OperatorTok{$}\NormalTok{Chromosome)}
\NormalTok{Geno2_melt_SNPinfo <-}\StringTok{ }\KeywordTok{arrange}\NormalTok{(Geno2_melt_SNPinfo, Chromosome)}

\KeywordTok{head}\NormalTok{(Geno2_melt_SNPinfo)}
\end{Highlighting}
\end{Shaded}

\begin{verbatim}
##   SNP_ID Sample_ID Group SNP_call Chromosome  Position
## 1  an1.4     SL-15 TRIPS      C/C          1 240498509
## 2  an1.4     SL-16 TRIPS      C/C          1 240498509
## 3  an1.4     SL-11 TRIPS      ?/?          1 240498509
## 4  an1.4     SL-12 TRIPS      C/C          1 240498509
## 5  an1.4     SL-18 TRIPS      C/C          1 240498509
## 6  an1.4      SL-2 TRIPS      C/C          1 240498509
\end{verbatim}

\begin{Shaded}
\begin{Highlighting}[]
\KeywordTok{tail}\NormalTok{(Geno2_melt_SNPinfo)}
\end{Highlighting}
\end{Shaded}

\begin{verbatim}
##             SNP_ID Sample_ID Group SNP_call Chromosome  Position
## 2612293 PZD00075.2    SYN262 ZMMIL      C/C         10 146599484
## 2612294 PZD00075.2     S0398 ZMMIL      ?/?         10 146599484
## 2612295 PZD00075.2     S1636 ZMMIL      ?/?         10 146599484
## 2612296 PZD00075.2    CU0201 ZMMIL      C/C         10 146599484
## 2612297 PZD00075.2     S0215 ZMMIL      C/C         10 146599484
## 2612298 PZD00075.2    CU0202 ZMMIL      C/C         10 146599484
\end{verbatim}

\begin{Shaded}
\begin{Highlighting}[]
\NormalTok{Group_SNP_counts <-}\StringTok{ }\NormalTok{Geno2_melt_SNPinfo }\OperatorTok
\StringTok{  }\KeywordTok{group_by}\NormalTok{ (Group, SNP_ID, SNP_call, Chromosome) }\OperatorTok
\StringTok{  }\KeywordTok{summarise}\NormalTok{ (}\DataTypeTok{row_counts =} \KeywordTok{n}\NormalTok{()) }
\KeywordTok{head}\NormalTok{(Group_SNP_counts, }\DataTypeTok{n=}\NormalTok{12L)}
\end{Highlighting}
\end{Shaded}

\begin{verbatim}
## # A tibble: 12 x 5
## # Groups:   Group, SNP_ID, SNP_call [12]
##    Group SNP_ID   SNP_call Chromosome row_counts
##    <chr> <fct>    <chr>         <int>      <int>
##  1 TRIPS abph1.20 ?/?               2         22
##  2 TRIPS abph1.22 ?/?               2         20
##  3 TRIPS abph1.22 A/A               2          2
##  4 TRIPS ae1.3    T/T               5         22
##  5 TRIPS ae1.4    ?/?               5          3
##  6 TRIPS ae1.4    A/A               5          1
##  7 TRIPS ae1.4    G/G               5         18
##  8 TRIPS ae1.5    ?/?               5          4
##  9 TRIPS ae1.5    T/T               5         18
## 10 TRIPS an1.4    ?/?               1          2
## 11 TRIPS an1.4    C/C               1         20
## 12 TRIPS ba1.6    ?/?               3          6
\end{verbatim}

\begin{Shaded}
\begin{Highlighting}[]
\NormalTok{Group_variableSNP <-}\StringTok{ }\NormalTok{Group_SNP_counts }\OperatorTok
\StringTok{  }\KeywordTok{group_by}\NormalTok{ (Group, SNP_ID, Chromosome) }\OperatorTok
\StringTok{  }\KeywordTok{summarise}\NormalTok{(}\DataTypeTok{SNP_call_types =} \KeywordTok{n}\NormalTok{())}
\KeywordTok{head}\NormalTok{(Group_variableSNP)}
\end{Highlighting}
\end{Shaded}

\begin{verbatim}
## # A tibble: 6 x 4
## # Groups:   Group, SNP_ID [6]
##   Group SNP_ID   Chromosome SNP_call_types
##   <chr> <fct>         <int>          <int>
## 1 TRIPS abph1.20          2              1
## 2 TRIPS abph1.22          2              2
## 3 TRIPS ae1.3             5              1
## 4 TRIPS ae1.4             5              3
## 5 TRIPS ae1.5             5              2
## 6 TRIPS an1.4             1              2
\end{verbatim}

\begin{Shaded}
\begin{Highlighting}[]
\NormalTok{VariableSNPs <-}\StringTok{ }\KeywordTok{filter}\NormalTok{(Group_variableSNP, SNP_call_types}\OperatorTok{>}\DecValTok{1}\NormalTok{)}
\KeywordTok{head}\NormalTok{(VariableSNPs, }\DataTypeTok{n=}\NormalTok{12L)}
\end{Highlighting}
\end{Shaded}

\begin{verbatim}
## # A tibble: 12 x 4
## # Groups:   Group, SNP_ID [12]
##    Group SNP_ID   Chromosome SNP_call_types
##    <chr> <fct>         <int>          <int>
##  1 TRIPS abph1.22          2              2
##  2 TRIPS ae1.4             5              3
##  3 TRIPS ae1.5             5              2
##  4 TRIPS an1.4             1              2
##  5 TRIPS ba1.6             3              3
##  6 TRIPS ba1.9             3              3
##  7 TRIPS bt2.5             4              2
##  8 TRIPS bt2.7             4              2
##  9 TRIPS Fea2.1            4              2
## 10 TRIPS lg2.2             3              2
## 11 TRIPS pbf1.1            2              2
## 12 TRIPS pbf1.2            2              2
\end{verbatim}

\begin{Shaded}
\begin{Highlighting}[]
\NormalTok{SNPbyChr <-}\StringTok{ }\NormalTok{VariableSNPs }\OperatorTok
\StringTok{  }\KeywordTok{group_by}\NormalTok{ (Chromosome, Group) }\OperatorTok
\StringTok{  }\KeywordTok{summarise}\NormalTok{(}\DataTypeTok{SNPcounts =}\KeywordTok{n}\NormalTok{())}

\NormalTok{SNPbyChr <-}\StringTok{ }\KeywordTok{arrange}\NormalTok{(SNPbyChr, Chromosome, }\KeywordTok{desc}\NormalTok{(SNPcounts))}
\NormalTok{SNPbyChr}
\end{Highlighting}
\end{Shaded}

\begin{verbatim}
## # A tibble: 160 x 3
## # Groups:   Chromosome [10]
##    Chromosome Group SNPcounts
##         <int> <chr>     <int>
##  1          1 ZMMIL       155
##  2          1 ZMMLR       155
##  3          1 ZMPBA       155
##  4          1 ZMPIL       153
##  5          1 ZMXCP       153
##  6          1 ZMXCH       149
##  7          1 ZMPJA       142
##  8          1 ZMMMR       121
##  9          1 TRIPS       105
## 10          1 ZLUXR       100
## # ... with 150 more rows
\end{verbatim}

\begin{Shaded}
\begin{Highlighting}[]
\NormalTok{SNPbyChr2 <-}\StringTok{ }\NormalTok{VariableSNPs }\OperatorTok
\StringTok{  }\KeywordTok{group_by}\NormalTok{(Chromosome) }\OperatorTok
\StringTok{  }\KeywordTok{summarise}\NormalTok{(}\DataTypeTok{SNPcountsAllGroup=}\KeywordTok{n}\NormalTok{())}
\NormalTok{SNPbyChr2}
\end{Highlighting}
\end{Shaded}

\begin{verbatim}
## # A tibble: 10 x 2
##    Chromosome SNPcountsAllGroup
##         <int>             <int>
##  1          1              1835
##  2          2              1534
##  3          3              1305
##  4          4              1028
##  5          5              1487
##  6          6               927
##  7          7              1169
##  8          8               727
##  9          9               681
## 10         10               645
\end{verbatim}

\section{Plots}\label{plots}

\section{SNPs per chromosome}\label{snps-per-chromosome}

Here we define SNPs are only the SNP varies withine a certain group.
Those monoallelic SNPs are excluded.

\begin{Shaded}
\begin{Highlighting}[]
\CommentTok{# To plot with all the SNPs in each chromosome, including all the groups}
\KeywordTok{ggplot}\NormalTok{(}\DataTypeTok{data =}\NormalTok{ SNPbyChr2) }\OperatorTok{+}\StringTok{ }\KeywordTok{geom_col}\NormalTok{ (}\DataTypeTok{mapping=}\KeywordTok{aes}\NormalTok{(}\DataTypeTok{x=}\NormalTok{Chromosome, }\DataTypeTok{y=}\NormalTok{SNPcountsAllGroup)) }\OperatorTok{+}\StringTok{ }\KeywordTok{scale_x_continuous}\NormalTok{(}\DataTypeTok{breaks =} \KeywordTok{c}\NormalTok{(}\DecValTok{1}\OperatorTok{:}\DecValTok{10}\NormalTok{))}
\end{Highlighting}
\end{Shaded}

\includegraphics{R_Assignment_Markdown_code_files/figure-latex/plot SNPs per chromosome-1.pdf}

\begin{Shaded}
\begin{Highlighting}[]
\KeywordTok{ggsave}\NormalTok{(}\StringTok{"../Output/SNPsPerChr.png"}\NormalTok{, }\DataTypeTok{width =}\DecValTok{6}\NormalTok{, }\DataTypeTok{height=}\DecValTok{5}\NormalTok{)}

\CommentTok{# To plot the number of SNPs from each group in each chromosome. Each group is color coded differently.}
\KeywordTok{ggplot}\NormalTok{(}\DataTypeTok{data =}\NormalTok{ SNPbyChr) }\OperatorTok{+}\StringTok{ }\KeywordTok{geom_point}\NormalTok{ (}\DataTypeTok{mapping=}\KeywordTok{aes}\NormalTok{(}\DataTypeTok{x=}\NormalTok{Chromosome, }\DataTypeTok{y=}\NormalTok{SNPcounts, }\DataTypeTok{color=}\NormalTok{ Group),}\DataTypeTok{size=}\DecValTok{4}\NormalTok{,}\DataTypeTok{alpha=}\FloatTok{0.7}\NormalTok{) }\OperatorTok{+}\StringTok{ }\KeywordTok{scale_x_continuous}\NormalTok{(}\DataTypeTok{breaks =} \KeywordTok{c}\NormalTok{(}\DecValTok{1}\OperatorTok{:}\DecValTok{10}\NormalTok{))}
\end{Highlighting}
\end{Shaded}

\includegraphics{R_Assignment_Markdown_code_files/figure-latex/plot SNPs per chromosome-2.pdf}

\begin{Shaded}
\begin{Highlighting}[]
\KeywordTok{ggsave}\NormalTok{(}\StringTok{"../Output/SNPsPerChrByGroup.png"}\NormalTok{, }\DataTypeTok{width =}\DecValTok{6}\NormalTok{, }\DataTypeTok{height=}\DecValTok{5}\NormalTok{)}
\end{Highlighting}
\end{Shaded}

This result shows us that different groups have different level of
multi-allelic SNPs. ZMMIL, ZMMLR, ZMPBA, ZMPIL, and ZMXCP tends to have
more multi-allelic SNPs than ZPERR, ZMXNT, ZDIPL, and ZMXNO.

\section{Missing data and amount of
heterozygosity}\label{missing-data-and-amount-of-heterozygosity}

\section{Reshape for heterozygous and missing
data}\label{reshape-for-heterozygous-and-missing-data}

\begin{Shaded}
\begin{Highlighting}[]
\CommentTok{# Create a function to convert Heterozygous to H, and missing to NA}
\NormalTok{hetNA <-}\StringTok{ }\ControlFlowTok{function}\NormalTok{ (x) \{}
  \ControlFlowTok{if}\NormalTok{ (x }\OperatorTok{==}\StringTok{ "A/A"}\OperatorTok{|}\StringTok{ }\NormalTok{x }\OperatorTok{==}\StringTok{"T/T"} \OperatorTok{|}\StringTok{ }\NormalTok{x }\OperatorTok{==}\StringTok{"C/C"}\OperatorTok{|}\StringTok{ }\NormalTok{x }\OperatorTok{==}\StringTok{"G/G"}\NormalTok{) \{}
    \KeywordTok{return}\NormalTok{(}\StringTok{"Homozygous"}\NormalTok{)}
\NormalTok{  \}}\ControlFlowTok{else} \ControlFlowTok{if}\NormalTok{ (x }\OperatorTok{==}\StringTok{ "?/?"}\NormalTok{)\{}
    \KeywordTok{return}\NormalTok{(}\StringTok{"NA"}\NormalTok{)}
\NormalTok{  \}}\ControlFlowTok{else}\NormalTok{\{}
    \KeywordTok{return}\NormalTok{(}\StringTok{"Heterozygous"}\NormalTok{)}
\NormalTok{  \}}
\NormalTok{\}}

\NormalTok{species <-}\StringTok{ }\ControlFlowTok{function}\NormalTok{ (x) \{}
  \ControlFlowTok{if}\NormalTok{(x}\OperatorTok{==}\StringTok{'ZMMIL'}\OperatorTok{|}\NormalTok{x}\OperatorTok{==}\StringTok{'ZMMLR'}\OperatorTok{|}\NormalTok{x}\OperatorTok{==}\StringTok{'ZMMMR'}\NormalTok{)\{}
    \KeywordTok{return}\NormalTok{(}\StringTok{"Maize"}\NormalTok{)}
\NormalTok{  \}}\ControlFlowTok{else} \ControlFlowTok{if}\NormalTok{(x}\OperatorTok{==}\StringTok{ 'ZMPBA'}\OperatorTok{|}\StringTok{ }\NormalTok{x}\OperatorTok{==}\StringTok{'ZMPIL'}\OperatorTok{|}\NormalTok{x}\OperatorTok{==}\StringTok{'ZMPJA'}\NormalTok{)\{}
    \KeywordTok{return}\NormalTok{(}\StringTok{"Teosinte"}\NormalTok{)}
\NormalTok{  \}}\ControlFlowTok{else}\NormalTok{ \{}
    \KeywordTok{return}\NormalTok{(}\StringTok{"Other"}\NormalTok{)}
\NormalTok{  \}}
\NormalTok{\}}

\CommentTok{# Create a new column with name HetNA}
\NormalTok{Geno2_melt_SNPinfo_HetNA <-}\StringTok{ }\NormalTok{Geno2_melt_SNPinfo}
\NormalTok{Geno2_melt_SNPinfo_HetNA}\OperatorTok{$}\NormalTok{HetNA <-}\StringTok{ }\KeywordTok{sapply}\NormalTok{(Geno2_melt_SNPinfo_HetNA}\OperatorTok{$}\NormalTok{SNP_call, hetNA)}
\NormalTok{Geno2_melt_SNPinfo_HetNA}\OperatorTok{$}\NormalTok{Species <-}\StringTok{ }\KeywordTok{sapply}\NormalTok{(Geno2_melt_SNPinfo_HetNA}\OperatorTok{$}\NormalTok{Group, species)}
\NormalTok{Geno2_melt_SNPinfo_HetNA <-}\StringTok{ }\KeywordTok{as.tibble}\NormalTok{(Geno2_melt_SNPinfo_HetNA)}
\NormalTok{Geno2_melt_SNPinfo_HetNA}\OperatorTok{$}\NormalTok{Sample_ID <-}\StringTok{ }\KeywordTok{as.character}\NormalTok{(Geno2_melt_SNPinfo_HetNA}\OperatorTok{$}\NormalTok{Sample_ID)}

\KeywordTok{tail}\NormalTok{(Geno2_melt_SNPinfo_HetNA, }\DataTypeTok{n=}\NormalTok{10L) }\CommentTok{# To check if the conversion was successful}
\end{Highlighting}
\end{Shaded}

\begin{verbatim}
## # A tibble: 10 x 8
##    SNP_ID     Sample_ID Group SNP_call Chromosome Position  HetNA  Species
##    <fct>      <chr>     <chr> <chr>         <int> <chr>     <chr>  <chr>  
##  1 PZD00075.2 S0351     ZMMIL C/C              10 146599484 Homoz~ Maize  
##  2 PZD00075.2 S1602     ZMMIL C/C              10 146599484 Homoz~ Maize  
##  3 PZD00075.2 WB0022a   ZMMIL C/C              10 146599484 Homoz~ Maize  
##  4 PZD00075.2 CU0200    ZMMIL C/C              10 146599484 Homoz~ Maize  
##  5 PZD00075.2 SYN262    ZMMIL C/C              10 146599484 Homoz~ Maize  
##  6 PZD00075.2 S0398     ZMMIL ?/?              10 146599484 NA     Maize  
##  7 PZD00075.2 S1636     ZMMIL ?/?              10 146599484 NA     Maize  
##  8 PZD00075.2 CU0201    ZMMIL C/C              10 146599484 Homoz~ Maize  
##  9 PZD00075.2 S0215     ZMMIL C/C              10 146599484 Homoz~ Maize  
## 10 PZD00075.2 CU0202    ZMMIL C/C              10 146599484 Homoz~ Maize
\end{verbatim}

\begin{Shaded}
\begin{Highlighting}[]
\NormalTok{Geno2_melt_SNPinfo_HetNA <-}\StringTok{ }\KeywordTok{arrange}\NormalTok{(Geno2_melt_SNPinfo_HetNA, Group, Sample_ID)}
\KeywordTok{head}\NormalTok{(Geno2_melt_SNPinfo_HetNA, }\DataTypeTok{n=}\NormalTok{10L)}
\end{Highlighting}
\end{Shaded}

\begin{verbatim}
## # A tibble: 10 x 8
##    SNP_ID      Sample_ID Group SNP_call Chromosome Position  HetNA Species
##    <fct>       <chr>     <chr> <chr>         <int> <chr>     <chr> <chr>  
##  1 an1.4       SL-10     TRIPS C/C               1 240498509 Homo~ Other  
##  2 id1.3       SL-10     TRIPS T/T               1 238902078 Homo~ Other  
##  3 PZA00010.5  SL-10     TRIPS ?/?               1 119703101 NA    Other  
##  4 PZA00013.10 SL-10     TRIPS ?/?               1 263276247 NA    Other  
##  5 PZA00013.11 SL-10     TRIPS C/C               1 263276005 Homo~ Other  
##  6 PZA00013.9  SL-10     TRIPS T/T               1 263276335 Homo~ Other  
##  7 PZA00017.1  SL-10     TRIPS G/G               1 35620725  Homo~ Other  
##  8 PZA00050.9  SL-10     TRIPS A/A               1 295459549 Homo~ Other  
##  9 PZA00081.17 SL-10     TRIPS ?/?               1 45675053  NA    Other  
## 10 PZA00098.14 SL-10     TRIPS C/C               1 161472433 Homo~ Other
\end{verbatim}

\begin{Shaded}
\begin{Highlighting}[]
\CommentTok{# To plot each sample with homozygous, heterozygous, and NA across SNP sites.}
\NormalTok{SampleHetCounts <-}\StringTok{ }\NormalTok{Geno2_melt_SNPinfo_HetNA }\OperatorTok
\StringTok{  }\KeywordTok{group_by}\NormalTok{(Sample_ID,HetNA) }\OperatorTok
\StringTok{  }\KeywordTok{summarise}\NormalTok{(}\DataTypeTok{HetNA_count=}\KeywordTok{n}\NormalTok{())}
\KeywordTok{head}\NormalTok{(SampleHetCounts)}
\end{Highlighting}
\end{Shaded}

\begin{verbatim}
## # A tibble: 6 x 3
## # Groups:   Sample_ID [2]
##   Sample_ID HetNA        HetNA_count
##   <chr>     <chr>              <int>
## 1 BKN_001   Heterozygous         115
## 2 BKN_001   Homozygous           710
## 3 BKN_001   NA                   114
## 4 BKN_002   Heterozygous          84
## 5 BKN_002   Homozygous           754
## 6 BKN_002   NA                   101
\end{verbatim}

\begin{Shaded}
\begin{Highlighting}[]
\KeywordTok{ggplot}\NormalTok{(}\DataTypeTok{data=}\NormalTok{SampleHetCounts) }\OperatorTok{+}\StringTok{ }\KeywordTok{geom_bar}\NormalTok{(}\DataTypeTok{mapping=}\KeywordTok{aes}\NormalTok{(}\DataTypeTok{x=}\NormalTok{Sample_ID, }\DataTypeTok{y=}\NormalTok{HetNA_count, }\DataTypeTok{fill=}\NormalTok{HetNA),}\DataTypeTok{stat =} \StringTok{"identity"}\NormalTok{)}
\end{Highlighting}
\end{Shaded}

\includegraphics{R_Assignment_Markdown_code_files/figure-latex/Reshape for heterpzygous and missing data-1.pdf}

\begin{Shaded}
\begin{Highlighting}[]
\KeywordTok{ggsave}\NormalTok{(}\StringTok{"../Output/SampleHetHomoNA.png"}\NormalTok{, }\DataTypeTok{width =}\DecValTok{10}\NormalTok{, }\DataTypeTok{height=}\DecValTok{5}\NormalTok{)}

\CommentTok{# To plot counts of each group with homozygous, heterozygous, and NA across SNP sites.}
\NormalTok{GroupHetCounts <-}\StringTok{ }\NormalTok{Geno2_melt_SNPinfo_HetNA }\OperatorTok
\StringTok{  }\KeywordTok{group_by}\NormalTok{(Group,HetNA) }\OperatorTok
\StringTok{  }\KeywordTok{summarise}\NormalTok{(}\DataTypeTok{HetNA_count=}\KeywordTok{n}\NormalTok{())}
\KeywordTok{head}\NormalTok{(GroupHetCounts)}
\end{Highlighting}
\end{Shaded}

\begin{verbatim}
## # A tibble: 6 x 3
## # Groups:   Group [2]
##   Group HetNA        HetNA_count
##   <chr> <chr>              <int>
## 1 TRIPS Heterozygous         637
## 2 TRIPS Homozygous         13610
## 3 TRIPS NA                  6411
## 4 ZDIPL Heterozygous         488
## 5 ZDIPL Homozygous         12111
## 6 ZDIPL NA                  1486
\end{verbatim}

\begin{Shaded}
\begin{Highlighting}[]
\KeywordTok{ggplot}\NormalTok{(}\DataTypeTok{data=}\NormalTok{GroupHetCounts) }\OperatorTok{+}\StringTok{ }\KeywordTok{geom_bar}\NormalTok{(}\DataTypeTok{mapping=}\KeywordTok{aes}\NormalTok{(}\DataTypeTok{x=}\NormalTok{Group, }\DataTypeTok{y=}\NormalTok{HetNA_count, }\DataTypeTok{fill=}\NormalTok{HetNA),}\DataTypeTok{stat =} \StringTok{"identity"}\NormalTok{)}
\end{Highlighting}
\end{Shaded}

\includegraphics{R_Assignment_Markdown_code_files/figure-latex/Reshape for heterpzygous and missing data-2.pdf}

\begin{Shaded}
\begin{Highlighting}[]
\KeywordTok{ggsave}\NormalTok{(}\StringTok{"../Output/GroupHetHomoNA.png"}\NormalTok{, }\DataTypeTok{width =}\DecValTok{10}\NormalTok{, }\DataTypeTok{height=}\DecValTok{5}\NormalTok{)}

\CommentTok{# To plot proportion of each group with homozygous, heterozygous, and NA across SNP sites}
\CommentTok{# Since the goem_bar with position=fill with the full data takes very long time and memory to run, I calculated the proporation seperately and used it as y input to get the proportion.}
\NormalTok{GroupTotal <-}\StringTok{ }\NormalTok{Geno2_melt_SNPinfo_HetNA }\OperatorTok
\StringTok{  }\KeywordTok{group_by}\NormalTok{(Group) }\OperatorTok
\StringTok{  }\KeywordTok{summarise}\NormalTok{(}\DataTypeTok{Grouptotal=}\KeywordTok{n}\NormalTok{())}
\KeywordTok{head}\NormalTok{(GroupTotal)}
\end{Highlighting}
\end{Shaded}

\begin{verbatim}
## # A tibble: 6 x 2
##   Group Grouptotal
##   <chr>      <int>
## 1 TRIPS      20658
## 2 ZDIPL      14085
## 3 ZLUXR      15963
## 4 ZMHUE       9390
## 5 ZMMIL     272310
## 6 ZMMLR    1179384
\end{verbatim}

\begin{Shaded}
\begin{Highlighting}[]
\NormalTok{GroupHetProp <-}\StringTok{ }\KeywordTok{merge}\NormalTok{(GroupHetCounts,GroupTotal,}\DataTypeTok{by=}\StringTok{"Group"}\NormalTok{)}
\NormalTok{GroupHetProp <-}\StringTok{ }\KeywordTok{mutate}\NormalTok{(GroupHetProp, }\DataTypeTok{Proportion =}\NormalTok{ (HetNA_count}\OperatorTok{/}\NormalTok{Grouptotal)}\OperatorTok{*}\DecValTok{100}\NormalTok{)}

\KeywordTok{ggplot}\NormalTok{(}\DataTypeTok{data=}\NormalTok{GroupHetProp) }\OperatorTok{+}\StringTok{ }\KeywordTok{geom_bar}\NormalTok{(}\DataTypeTok{mapping=}\KeywordTok{aes}\NormalTok{(}\DataTypeTok{x=}\NormalTok{Group, }\DataTypeTok{y=}\NormalTok{Proportion, }\DataTypeTok{fill=}\NormalTok{HetNA),}\DataTypeTok{stat =} \StringTok{"identity"}\NormalTok{)}
\end{Highlighting}
\end{Shaded}

\includegraphics{R_Assignment_Markdown_code_files/figure-latex/Reshape for heterpzygous and missing data-3.pdf}

\begin{Shaded}
\begin{Highlighting}[]
\KeywordTok{ggsave}\NormalTok{(}\StringTok{"../Output/GroupHetHomoNAProp.png"}\NormalTok{, }\DataTypeTok{width =}\DecValTok{10}\NormalTok{, }\DataTypeTok{height=}\DecValTok{5}\NormalTok{)}
\end{Highlighting}
\end{Shaded}

\section{My own visualization}\label{my-own-visualization}

I would like to see if the heterozygous status across the positions in
Chromosome 1, and how they are different for each group.

\begin{Shaded}
\begin{Highlighting}[]
\NormalTok{Geno2_melt_SNPinfo_HetNA}\OperatorTok{$}\NormalTok{Position <-}\StringTok{ }\KeywordTok{as.integer}\NormalTok{(Geno2_melt_SNPinfo_HetNA}\OperatorTok{$}\NormalTok{Position)}
\NormalTok{Chr1SNP <-}\StringTok{ }\NormalTok{Geno2_melt_SNPinfo_HetNA }\OperatorTok
\StringTok{  }\KeywordTok{filter}\NormalTok{(Chromosome }\OperatorTok{==}\StringTok{ }\DecValTok{1}\NormalTok{) }\OperatorTok
\StringTok{  }\KeywordTok{arrange}\NormalTok{(Position) }\OperatorTok
\StringTok{  }\KeywordTok{group_by}\NormalTok{(Group,HetNA,Position) }\OperatorTok
\StringTok{  }\KeywordTok{summarise}\NormalTok{(}\DataTypeTok{n_rows=}\KeywordTok{n}\NormalTok{())}

\KeywordTok{ggplot}\NormalTok{(}\DataTypeTok{data=}\NormalTok{Chr1SNP)}\OperatorTok{+}\KeywordTok{geom_density}\NormalTok{(}\DataTypeTok{mapping=}\KeywordTok{aes}\NormalTok{(}\DataTypeTok{x=}\NormalTok{Position,}\DataTypeTok{fill=}\NormalTok{HetNA),}\DataTypeTok{alpha=}\FloatTok{0.5}\NormalTok{) }\OperatorTok{+}\StringTok{ }\KeywordTok{facet_wrap}\NormalTok{(}\OperatorTok{~}\NormalTok{Group)}
\end{Highlighting}
\end{Shaded}

\includegraphics{R_Assignment_Markdown_code_files/figure-latex/SNP state across Chr1 in different group-1.pdf}

\begin{Shaded}
\begin{Highlighting}[]
\KeywordTok{ggsave}\NormalTok{(}\StringTok{"../Output/Chr1SNP.png"}\NormalTok{, }\DataTypeTok{width =}\DecValTok{10}\NormalTok{, }\DataTypeTok{height=}\DecValTok{5}\NormalTok{)}
\end{Highlighting}
\end{Shaded}

Note that the \texttt{echo\ =\ FALSE} parameter was added to the code
chunk to prevent printing of the R code that generated the plot.


\end{document}
